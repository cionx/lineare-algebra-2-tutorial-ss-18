\section{}

Wir haben im Tutorium gesehen, dass ein Endomorphismus $f \colon V \to V$ eines endlichdimensionalen $K$-Vektorraums $V$ genau dann eine Jordan-Chevalley-Zerlegung besitzt, wenn $f$ eine Jordan-Normalform besitzt.
Ingesamt sind somit für $f$ die folgenden Bedingungen äquivalent:

\begin{enumerate}
  \item
    Das charakteristische Polynom $\charpol{f}$ zerfällt in Linearfaktoren.
  \item
    Das Minimalpolynom $\minpol{f}$ zerfällt in Linearfaktoren.
  \item
    Es gilt die Hauptraumzerlegung $V = \bigoplus_{\lambda \in K} \hauptspace{f}{\lambda}$.
  \item
    Der Endomorphismus $f$ besitzt eine Jordan-Normalform.
  \item
    Der Endomorphismus $f$ besitzt eine Jordan-Chevalley-Zerlegung.
\end{enumerate}

Die Jordan-Chevalley-Zerlegung hat einige Vorteile gegenüber der üblichen Jordan-Normalform:



\subsection*{Koordinatenfreiheit}

Die Jordan-Chevalley-Zerlegung lässt sich als eine koordinatenfreie Version der Jordan-Normalform verstehen:
Um die Jordan-Normalform von $f$ zu betrachten und zu definieren, müssen wir eine Basis $B$ von $V$ wählen, um $\repmatrixendo{f}{B}$ zu betrachten.
Die Jordan-Chevalley-Zerlegung hat diese Einschränkung nicht.



\subsection*{Polynomielle Abhängigkeit von $f_d$ und $f_n$}

Existiert die Jordan-Chevalley-Zerlegung $f = f_d + f_n$, so gibt es Polynome $p, q \in K[X]$ mit $f_d = p(f)$ und $f_n = q(f)$.
Der diagonalisierbare sowie nilpotente Anteil von $f$ ist also ein Polynom in $f$.
Dies hat in der Praxis nützliche Konsequenzen:
So kommutiert etwa ein Endomorphismus $g \colon V \to V$ genau dann mit $f$, wenn er mit $f_d$ und $f_n$ kommutiert.

(Dies wird häufig benutzt, um die Eindeutigkeit der Jordan-Chevalley-Zerlegung zu zeigen:
Ist $f = f'_d + f'_n$ eine weitere Zerlegung mit $f'_d$ diagonalisierbar, $f'_n$ nilpotent und $f'_d, f'_n$ kommutierend, so kommutieren $f'_n$ und $f'_d$ mit $f$, und somit auch mit $f_d$ und $f_n$.
Dann sind $f_d$ und $f'_d$ simultan diagonalisierbar, und somit ist auch $f_d - f'_d$ diagonalisierbar.
Außerdem ist dann auch $f'_n - f_n$ nilpotent.
Der Endomorphismus $f_d - f'_d = f'_n - f_n$ ist also sowohl diagonalisierbar als auch nilpotent.
Es muss deshalb $f_d - f'_d = f'_n - f_n = 0$ gelten, und somit $f_d = f'_d$ und $f_n = f'_n$.)



\subsection*{Verallgemeinerung auf mehr Endomorphismen}

Ein $f$-invarianter Untervektorraum $U \subseteq V$ heißt $f$-\emph{irreduzibel}, bzw.\ $f$-\emph{einfach}, wenn $U \neq 0$ gilt, und $0$ und $U$ die einzigen $f$-invarianten Untervektorräume von $U$ sind.
Es lässt sich zeigen, dass für $f$ die folgenden Bedingungen äquivalent sind:

\begin{enumerate}
  \item
    Es gilt $V = U_1 \oplus \dotsb \oplus U_n$ für $f$-invariante, $f$-irreduzible Untervektorräume $U_i \subseteq V$.
  \item
    Es gilt $V = U_1 + \dotsb + U_n$ für $f$-invariante, $f$-irreduzible Untervektorräume $U_i \subseteq V$.
  \item
    Für jeden $f$-invarianten Untervektorraum $U \subseteq V$ gibt es einen $f$-invarianten Untervektorraum $W \subseteq V$ mit $V = U \oplus W$.
\end{enumerate}

Erfüllt $f$ eine (und somit alle) der obigen Bedingungen, so heißt $f$ \emph{halbeinfach}.
Zerfällt $\charpol{f}$ in Linearfaktoren, so ist $f$ genau dann halbeinfach, wenn $f$ diagonalisierbar ist;
diese Äquivalenz gilt inbesondere dann, wenn $K$ algebraisch abgeschlossen ist.
(Allgemeiner ist $f$ genau dann halbeinfach, wenn kein Primfaktor des Minimalpolyoms $\minpol{f}$ mehrfach vorkommt.
Zerfällt $\charpol{f}$, und somit auch $\minpol{f}$, in Linearfaktoren, so gilt dies genau dann, wenn $\minpol{f}$ in \emph{paarweise verschiedene} Linearfaktoren zerfällt, wenn also $f$ diagonalisierbar ist.)

Für bestimmte Arten von Körpern lässt sich die Jordan-Chevalley-Zerlegung nun passend Verallgemeinern, indem man \emph{diagonalisierbar} durch \emph{halbeinfach} ersetzt:

\begin{theorem}[Jordan-Chevalley-Zerlegung]
  Ist $K$ ein perfekter Körper\footnote{
  Ein Körper $K$ ist \emph{perfekt}, wenn die irreduziblen Polynom aus $K[X]$ über dem algebraische Abschluss $\overline{K}$ keine mehrfachen Nullstellen haben. 
  Diese Bedingung erlaubt es, diese Verallgemeinerne Jordan-Chevalley-Zerlegung auf den bereits bekannten Sonderfall zurückzuführen.}
  und $f \colon V \to V$ ein Endomorphismus eines endlichdimensionalen $K$-Vektorraums, so gibt es eindeutige Endomorphismes $f_h, f_n \colon V \to V$ so dass
  \begin{enumerate}
    \item
      $f = f_h + f_n$ gilt,
    \item
      $f_h$ halbeinfach und $f_n$ nilpotent ist,
    \item
      $f_h$ und $f_n$ kommutieren.
  \end{enumerate}
\end{theorem}



\subsection*{Verallemeinerung auf halbeinfache Lie-Algebren}

Ist $\mathfrak{g}$ eine endlichdimensionale, halbeinfache, komplexe Lie-Algebra, so gibt es \emph{halbeinfache} Element $s \in \mathfrak{g}$ und \emph{nilpotente} Elemente $n \in \mathfrak{g}$.
Für jedes $x \in \mathfrak{g}$ gibt es dann eindeutige $s, n \in \mathfrak{g}$, so dass $x = s + n$ gilt, $s$ halbeinfach und $n$ nilpotent ist, und $s$ und $n$ kommutieren.



\subsection*{Verallgemeinerung auf $\GL{n}{K}$}

Ein Endomorphismus $u \colon V \to V$ heißt \emph{unipotent}, wenn $u - \id_V$ nilpotent ist\footnote{
Ein Endomorphismus $g \colon V \to V$ heißt \emph{$\lambda$-potent} für $\lambda \in K$ wenn $g - \lambda \id_V$ nilpotent ist.
Inbesondere ist \enquote{$0$-potent} genau \enquote{nilpotent}, und \enquote{$1$-potent} genau \enquote{unipotent}.}.
Die bisher bekannte \emph{additive Jordan-Chevalley-Zerlegung} führt zur \emph{multiplikativen Jordan-Chevalley-Zerlegeng}:

\begin{theorem}[Multiplikative Jordan-Chevalley-Zerlegung]
  Es sei $f \in \GL{}{V}$, so dass $\charpol{f}$ in Linearfaktoren zerfällt.
  Dann gibt es eindeutige $f_d, f_u \in \GL{}{V}$, so dass
  \begin{enumerate}
    \item
      $f = f_d f_u$ gilt,
    \item
      $f_d$ diagonalisierbar und $f_u$ unipotent ist,
    \item
      $f_d$ und $f_u$ kommutieren.
  \end{enumerate}
\end{theorem}

Hat $f \in \GL{}{V}$ die additive Jordan-Chevalley-Zerlegung $f = f_d + f_n$ und die multiplikative Jordan-Chevalley-Zerlegung $f = f'_d f'_u$, so gilt auch $f_d \in \GL{}{V}$, und die beiden Zerlegungen hängen durch
\[
  f'_d = f_d
  \qquad\text{und}\qquad
  f'_u = 1 + f_d^{-1} f_n
\]
zusammen.
Die multiplikative Jordan-Zerlegungen ist etwa im Kontext linearer algebraischer Gruppen von Bedeutung.
