\section{}

Wir lösen das Problem für eine spezielle Körper:


\subsection*{Der Fall $K = \Real$}

Wir betrachten zunächst den Fall, der in der korrigierten Augabenstellung zu finden ist.
Wir betrachten den Untervektorraum $U \subseteq \matrices{n}{\Real}$ der echten oberen Dreiecksmatrizen.
Für alle $A \in U$ gilt
\[
    s(A,A)
  = \tr(A^2)
  = 0
\]
da auch $A^2$ eine echte obere Dreiecksmatrix ist.
Wir zeigen nun, dass $U$ bereits ein dimensionsmaximaler (und somit auch inklusionsmaximaler) total isotoper Untervektorraum von $V$ bezüglich $s$ ist.

Hierfür betrachten wir den Untervektorraum der symmetrischen Matrizen $S \subseteq \matrices{n}{\Real}$.
Für alle $A \in S$ gilt
\[
    s(A,A)
  = \tr(A^2)
  = \tr(A A^T)
  = \sum_{i=1}^n (A A^T)_{ii}
  = \sum_{i,j=1}^n A_{ij} A^T_{ji}
  = \sum_{i,j=1}^n A_{ij}^2 \,,
\]
und somit $s(A,A) > 0$ für $A \neq 0$.
Das zeigt, dass $\restrict{s}{S}$ positiv definit ist.

Für den total isotopen Untervektorraum $U' \subseteq \matrices{n}{\Real}$ muss deshalb $U' \cap S = 0$ gelten.
Es ergibt sich dann, dass
\[
        \dim(S) + \dim(U')
  =     \dim(S + U') + \dim(S \cap U')
  =     \dim(S + U')
  \leq  \dim(\matrices{n}{\Real})
  =     n^2
\]
gilt, und somit
\[
        \dim(U') 
  \leq  n^2 - \dim(S) \,.
\]
Es gilt $\dim(S) = n(n+1)/2$, denn eine Basis von $S$ ist durch die Matrizen $E_{ii}$ mit $i = 1, \dotsc, n$ zusammen mit den Matrizen $E_{ij} + E_{ji}$ mit $i < j$ gegeben.
Es ergibt sich somit, dass
\[
        \dim(U')
  \leq  n^2 - \frac{n(n+1)}{2}
  =     \frac{n(n-1)}{2}
\]
für jeden total isotropen Untervektorraum $U' \subseteq \matrices{n}{\Real}$ gilt.
Es gilt $\dim(U) = n(n-1)/2$, also ist $U$ ein dimensionsmaximaler total isotroper Untervektorraum von $\matrices{n}{\Real}$ bezüglich $s$.





\subsection*{Der Fall $\ringchar(K) = 2$}

Gilt $\ringchar(K) = 2$ so gilt für alle $A \in \matrices{n}{K}$, dass
\begin{align*}
      s(A,A)
  &=  \tr(A^2)
   =  \sum_{i=1}^n (A^2)_{ii}
   =  \sum_{i,j=1}^n A_{ij} A_{ji}
  \\
  &=    \sum_{i=1}^n A_{ii}^2
      + \sum_{i < j} A_{ij} A_{ji}
      + \sum_{i > j} A_{ij} A_{ji}
   =    \sum_{i=1}^n A_{ii}^2
      + \sum_{i < j} A_{ij} A_{ji}
      + \sum_{i < j} A_{ji} A_{ij}
  \\
  &=    \sum_{i=1}^n A_{ii}^2
      + 2\sum_{i < j} A_{ij} A_{ji}
   =  \sum_{i=1}^n A_{ii}^2
   =  \left( \sum_{i=1}^n A_{ii} \right)^2
   =  \tr(A)^2 \,.
\end{align*}
Es ist deshalb
\[
            V
  \defined  \ker(\tr)
\]
der eindeutige dimensions- und inklusionsmaximale total isotope Unterraum von $\matrices{n}{K}$ bezüglich $s$.
Man bemerke, dass bereits $\dim V = n^2 - 1$ gilt.





\subsection*{Eine Bemerkung zum Fall $\ringchar(K) \neq 2$}

Im Fall $\ringchar(K) \neq 2$ kann jeder total isotrope Untervektrraum $U \subseteq \matrices{n}{\Real}$ höchstens dimension $n^2/2$ haben.
Für die Einschränkung $\restrict{s}{U}$ gilt dann nämlich $\restrict{s}{U}(u,u) = 0$ für alle $u \in U$, und wegen der Polarisationsform
\[
    s(u_1, u_2)
  = \frac{s(u_1 + u_2, u_1 + u_2) - s(u_1, u_2) - s(u_2,u_2)}{2}
\]
somit bereits $\restrict{s}{U}(u_1, u_2) = 0$ für alle $u_1, u_2 \in U$, also $\restrict{s}{U} = 0$.
Dies bedeutet, dass $U \subseteq U^\perp$ gilt.
Da $s$ regulär ist, gilt
\[
    \dim U^\perp
  = \dim \matrices{n}{\Real} - \dim U
  = n^2 - \dim U \,.
\]
Es muss also
\[
        \dim U
  \leq  \dim U^\perp
  \leq  n^2 - \dim U
\]
gelten, und somit
\[
        2 \dim U
  \leq  n^2 \,.
\]


\subsection*{Der Fall $\ringchar(K) \neq 2$, $K$ quadratisch abgeschlossen}

Ist $K$ quadratisch abgeschlossen, so gibt es ein Element $i \in K$ mit $i^2 = -1$.
Wir können dann den Untervektorraum $U \subseteq \matrices{n}{K}$ aller oberen Dreiecksmatrizen der Form
\[
  \begin{pmatrix}
    a_1 & *     & \cdots  & \cdots  & \cdots  & *       \\
        & i a_1 & \ddots  &         &         & \vdots  \\
        &       & a_2     & \ddots  &         & \vdots  \\
        &       &         & i a_2   & \ddots  & \vdots  \\
        &       &         &         & \ddots  & *       \\
        &       &         &         &         & ?
  \end{pmatrix}
\]
mit $a_1, a_2, \dotsc \in K$ betrachten, wobei der letzte Diagonaleintrag $?$ entweder $0$ oder $i a_{n/2}$ ist, je nach Parität von $n$.
Es ist dann $U$ ein total isotroper Untvervektorraum von $\matrices{n}{K}$ bezüglich $K$, denn für die obige Matrix $A$ gilt
\[
    s(A,A)
  = \tr(A^2)
  = \sum_j (A^2)_{jj}
  = \sum_j \left( a_j^2 + (i a_j)^2 \right)
  = \sum_j ( a_j^2 - a_j^2 )
  = 0 \,.
\]
Außerdem gilt $\dim U = \lfloor n^2/2 \rfloor$, weshalb $U$ nach der vorherigen Bemerkung bereits dimensionsmaximal unter allen total isotropen Untervektorräumen von $\matrices{n}{K}$ ist.



