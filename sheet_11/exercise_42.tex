\section{}





\subsection{}

Es lässt sich beispielsweise $v = \columnrvector{1 \\ 0 \\ -1}$ wählen.





\addtocounter{subsection}{1}
\subsection{}



\subsubsection*{Simultanes Zeilen- und Spaltenumformen}
Häufig ergibt sich das Problem, dass eine Matrix $A \in \matrices{n}{K}$ in eine gewisse zu $A$ kongruente Normalenform (Diagonalform, symplektische Normalenform, \dots) gebracht werden soll, und zudem eine Matrix $S \in \GL{n}{K}$ bestimmt werden soll, für welche $S^T A S$ die gewünschte Normalenform hat.
Wir geben nun ein allgemeines Verfahren an, wie sich an dieses Problem herangehen lässt.

Die Grundidee besteht darin, die Matrix $A$ durch Zeilen- und Spaltenumformungen in die gewünschte Normalenform $N$ zu bringen.
Das Problem dabei besteht darin, dass Zeilen- und Spaltenumformungen kongruenzzerstörend sind, d.h.\ wenn aus $A$ durch Anwenden einer elementaren Zeilen- oder Spaltenumformung eine Matrix $A'$ entsteht, so sind $A'$ und $A$ im Allgemeinen nicht mehr kongruent zueinander.

Dieses Problem lässt sich mithilfe \emph{simultaner Zeilen- und Spaltenumformungen} lösen:
Nach jeder elementaren Zeilenumformung (bzw.\ Spaltenumformung) führen wir anschließend auch die entsprechende elementare Spaltenumformung (bzw.\ Zeilenumformung) durch.
Das Anwenden einer elementaren Zeilenumformung auf $A$ entspricht dem Multiplizieren mit einer Elementarmatrix $E$ von links, und das anschließende Anwenden der entsprechenden Spaltenumformung entspricht der anschließenden Multiplikation mit $E^T$ von rechts.
Die entstehende Matrix $E A E^T$ ist dann kongruent zu $A$.
\[
  \begin{tikzcd}[column sep = 7em]
      A
      \arrow{r}[above]{\text{ZU}}
      \arrow[bend left]{r}[above]{\text{i.A.\ nicht kongruent}}
      \arrow[bend right]{rr}[below]{\text{kongruent}}
    & EA
      \arrow{r}[above]{\text{SU}}
    & EAE^T
  \end{tikzcd}
\]

Wenn wir also versuchen, $A$ durch simultane Zeilen- und Spaltenumformungen in die gewünschte Normalenform $N$ zu bringen, so werden $A$ und $N$ kongruent sein.
\[
  \begin{tikzcd}[column sep = small]
      A
      \arrow{r}
    & E_1 A E_1^T
      \arrow{r}
    & E_2 E_1 A E_1^T E_2^T
      \arrow{r}
    & \cdots
      \arrow{r}
    & E_n \dotsm E_1 A E_1^T \dotsm E_n^T
      \arrow[equal]{r}
    & N
  \end{tikzcd}
\]

Es lässt sich dann auch eine Matrix $S \in \GL{n}{K}$ mit $S^T A S = N$ bestimmen, denn für $S = E_1^T \dotsm E_n^T$ gilt
\[
    N
  = E_n \dotsm E_1 A E_1^T \dotsm E_n^T
  = S^T A S \,. 
\]
Die Matrix $S$ lässt sich ausrechnen, indem man die genutzen Spaltenumformungen auf die Einheitsmatrix $I$ anwendet:
\[
  \begin{tikzcd}
      I
      \arrow{r}[above]{\text{SU}}
    & I E_1^T
      \arrow{r}[above]{\text{SU}}
    & I E_1^T E_2^T
      \arrow{r}[above]{\text{SU}}
    & \cdots
      \arrow{r}[above]{\text{SU}}
    & I E_1^T \dotsm E_n^T
      \arrow[equal]{r}
    & S
  \end{tikzcd}
\]



\subsubsection*{Sonderfall: Symmetrische Matrizen}

Es sei nun $A \in \matrices{n}{K}$ symmetrisch, wobei $\ringchar(K) \neq 2$ gilt.
Wir zeigen nun, wie sich mit dem obigen Verfahren eine Diagonalmatrix $D \in \matrices{n}{K}$ und eine Matrix $S \in \GL{n}{K}$ bestimmen lassen, so dass $S^T A S = D$ gilt.
Hierfür reicht es, die Matrix $A$ durch simultanes Zeilen- und Spaltenumformen in eine Diagonalform $D$ zu bringen;
durch Anwenden der genutzen Spaltenumformungen auf die Einheitsmatrix ergibt sich dann eine Matrix $S \in \GL{n}{K}$ mit $S^T A S = D$.

Um die Matrix $A$ durch simultanes Zeilen- und Spaltenumformen in eine Diagonalform $D$ zu bringen, lässt sich ähnlich zum Gauß-Algorithmus vorgehen.
\begin{enumerate}
  \item 
    Ist die Matrix $A$ bereits in der Form
    \[
        A
      = \begin{pmatrix}
          *       & 0       & \cdots  & 0       \\
          0       & *       & \cdots  & *       \\
          \vdots  & \vdots  & \ddots  & \vdots  \\
          0       & *       & \cdots  & *
        \end{pmatrix}
    \]
    so lässt sich induktiv vorgehen.
  \item
    Falls $A$ von der Form
    \[
        A
      = \begin{pmatrix}
          a_1     & a_2     & \cdots  & a_n     \\
          a_2     & *       & \cdots  & *       \\
          \vdots  & \vdots  & \ddots  & \vdots  \\
          a_n     & *       & \cdots  & *
        \end{pmatrix}
    \]
    ist, und $a_1 \neq 0$ gilt, so zieht man für alle $i = 2, \dotsc, n$ das $a_i/a_1$-fache der ersten Zeile von der $i$-ten Zeile ab, sowie jeweils anschließend auch das $a_i/a_1$-fache der ersten Spalte von der $i$-ten Spalte.
    (Alternativ lässt sich auch die $i$-te Zeile (bzw.\ Spalte) mit $a_1$ multiplizieren, um anschließend das $a_i$-fache der ersten Zeile (bzw.\ Spalte) von der $i$-ten Zeile (bzw.\ Spalte) abzuziehen.)
    Hierdurch erhalten wir eine Matrix der Form
    \[
      \begin{pmatrix}
        a_1     & 0       & \cdots  & 0       \\
        0       & *       & \cdots  & *       \\
        \vdots  & \vdots  & \ddots  & \vdots  \\
        0       & *       & \cdots  & *
      \end{pmatrix},
    \]
    und befinden uns im ersten Fall.
  \item
    Falls $A$ von der Form
    \[
      \begin{pmatrix}
        0       & *       & \cdots  & *       & a_i     & *       & \cdots  & *       \\
        *       & *       & \cdots  & \cdots  & \cdots  & \cdots  & \cdots  & *       \\
        \vdots  & \vdots  & \ddots  &         &         &         &         & \vdots  \\
        *       & \vdots  &         & \ddots  &         &         &         & \vdots  \\
        a_i     & \vdots  &         &         & \ddots  &         &         & \vdots  \\
        *       & \vdots  &         &         &         & \ddots  &         & \vdots  \\
        \vdots  & \vdots  &         &         &         &         & \ddots  & \vdots  \\
        *       & *       & \cdots  & \cdots  & \cdots  & \cdots  & \cdots  & *
      \end{pmatrix}
    \]
    ist, so dass $a_i \neq 0$ gilt, so addieren wir die $i$-te Zeile auf die erste Zeile, sowie anschließend auch die $i$-te Spalte auf die erste Spalten.
    Wir erhalten dann eine Matrix der Form
    \[
      \begin{pmatrix}
        2a_i    & *       & \cdots  & *       \\
        *       & *       & \cdots  & *       \\
        \vdots  & \vdots  & \ddots  & \vdots  \\
        *       & *       & \cdots  & *
      \end{pmatrix}
    \]
    und befinden uns im zweiten Fall (es gilt $2 a_i \neq 0$ da $a_i \neq 0$ und $\ringchar(K) \neq 2$ gelten).
    Alternativ lässt sich auch das $1/2$-fache der $i$-ten Zeile (bzw.\ Spalte) auf die erste Zeile (bzw.\ Spalte) addieren.
    Hierdurch erhalten wir dann eine Matrix der Form
    \[
      \begin{pmatrix}
        a_i     & *       & \cdots  & *       \\
        *       & *       & \cdots  & *       \\
        \vdots  & \vdots  & \ddots  & \vdots  \\
        *       & *       & \cdots  & *
      \end{pmatrix}.
    \]
\end{enumerate}


\addtocounter{subsection}{-2}
\subsection{}

Wir führen den obigen Algorithmus für die gegeben Matrix aus:
\begin{align*}
  &\,
  \begin{pmatrix}
    2 & 1 & 3 \\ 
    1 & 2 & 0 \\
    3 & 0 & 1 \\
  \end{pmatrix}
  \xlongrightarrow{2 Z_2}
  \begin{pmatrix}
    2 & 1 & 3 \\ 
    2 & 4 & 0 \\
    3 & 0 & 1 \\
  \end{pmatrix}
  \\
  \xlongrightarrow{2 S_2}&\,
  \begin{pmatrix}
    2 & 2 & 3 \\ 
    2 & 8 & 0 \\
    3 & 0 & 1 \\
  \end{pmatrix}
  \xlongrightarrow{Z_2 - Z_1}
  \begin{pmatrix*}[r]
    2 & 2 &  3  \\ 
    0 & 6 & -3  \\
    3 & 0 &  1  \\
  \end{pmatrix*}
  \\
  \xlongrightarrow{S_2 - S_1}&\,
  \begin{pmatrix*}[r]
    2 &  0  &  3  \\ 
    0 &  6  & -3  \\
    3 & -3  &  1  \\
  \end{pmatrix*}
  \xlongrightarrow{2 Z_3}
  \begin{pmatrix*}[r]
    2 &  0  &  3  \\ 
    0 &  6  & -3  \\
    6 & -6  &  2  \\
  \end{pmatrix*}
  \\
  \xlongrightarrow{2 S_3}&\,
  \begin{pmatrix*}[r]
    2 &  0  &  6  \\ 
    0 &  6  & -6  \\
    6 & -6  &  4  \\
  \end{pmatrix*}
  \xlongrightarrow{Z_3 - 3 Z_1}
  \begin{pmatrix*}[r]
    2 &  0  &   6 \\ 
    0 &  6  &  -6 \\
    0 & -6  & -14 \\
  \end{pmatrix*}
  \\
  \xlongrightarrow{S_3 - 3 S_1}&\,
  \begin{pmatrix*}[r]
    2 &  0  &   0 \\ 
    0 &  6  &  -6 \\
    0 & -6  & -14 \\
  \end{pmatrix*}
  \xlongrightarrow{Z_3 + Z_2}
  \begin{pmatrix*}[r]
    2 & 0 &   0 \\ 
    0 & 6 &  -6 \\
    0 & 0 & -20 \\
  \end{pmatrix*}
  \\
  \xlongrightarrow{S_3 + S_2}&\,
  \begin{pmatrix*}[r]
    2 & 0 &   0 \\ 
    0 & 6 &   0 \\
    0 & 0 & -20 \\
  \end{pmatrix*}
  \eqqcolon
  D.
\end{align*}
Man bemerke, dass nach jeder \enquote{simultanen} Zeilen- und Spaltenumformung die entstandene Matrix wieder symmetrisch ist.

Wir bestimmen nun eine Matrix $S \in \GL{3}{\Real}$ mit $S^T A S = D$ indem wir die verwendeten Spaltenumformungen auf die Einheitsmatrix anwenden
\begin{align*}
  \begin{pmatrix*}[r]
    1 & 0 & 0 \\ 
      & 1 & 0 \\
      &   & 1 \\
  \end{pmatrix*}
  \xlongrightarrow{2 S_2}
  \begin{pmatrix*}[r]
    1 & 0 & 0 \\ 
      & 2 & 0 \\
      &   & 1 \\
  \end{pmatrix*}
  &\xlongrightarrow{S_2 - S_1}
  \begin{pmatrix*}[r]
    1 & -1  & 0 \\ 
      &  2  & 0 \\
      &     & 1 \\
  \end{pmatrix*}
  \xlongrightarrow{2 S_3}
  \begin{pmatrix*}[r]
    1 & -1  & 0 \\ 
      &  2  & 0 \\
      &     & 2 \\
  \end{pmatrix*}
  \\
  &\xlongrightarrow{S_3 - 3 S_1}
  \begin{pmatrix*}[r]
    1 & -1  & -3  \\ 
      &  2  &  0  \\
      &     &  2  \\
  \end{pmatrix*}
  \xlongrightarrow{S_3 + S_2}
  \begin{pmatrix*}[r]
    1 & -1  & -4  \\ 
      &  2  &  2  \\
      &     &  2  \\
  \end{pmatrix*}
  \eqqcolon
  S \,.
\end{align*}





\subsection{}

Tatsächlich ist bereits jede reelle symmetrische Matrix diagonalisierbar (dies wird in der Vorlesung noch gezeigt werden).
Wir werden den Zwischenwertsatz aus der Analysis nutzen, um die Diagonalisierbarkeit von $A$ zu begründen:
Das charakteristische Polynom von $A$ ist
\[
    \charpol{A}
  = X^3 - 5 X^2 - 2 X + 15 \,.
\]
Es gilt deshalb
\begin{itemize}
  \item
    $\charpol{A}(x) \to -\infty$ für $x \to -\infty$,
  \item
    $\charpol{A}(0) = 15 > 0$,
  \item
    $\charpol{A}(2) = -1 < 0$,
  \item
    $\charpol{A}(x) \to \infty$ für $x \to \infty$.
\end{itemize}
Aus dem Zwischenwertsatz ergibt sich, dass $\charpol{A}$ in jeder der Intervalle $(-\infty,0)$, $(0, 2)$ und $(2,\infty)$ eine Nullstelle besitzt.
Somit besitzt $\charpol{A}$ drei verschiedene Nullstellen, zerfällt also in drei verschiedene Linearfaktoren.
Die Matrix $A$ ist somit diagonalisierbar.






