\section{}





\addtocounter{subsection}{1}
\subsection{}

Wir beschreiben, wie sich für einen Körper $K$ mit $\ringchar(K) \neq 2$ eine schiefsymmetrische Matrix $A \in \matrices{n}{K}$ durch simultane Zeilen- und Spaltenumformungen in symplektische Normalenform bringen lässt.
Hierfür bringen wir die Matrix $A$ zunächst in die Form
\[
  \begin{pmatrix*}[r]
     0  & 1 &         &     &   &   &         &   \\
    -1  & 0 &         &     &   &   &         &   \\
        &   & \ddots  &     &   &   &         &   \\
        &   &         &  0  & 1 &   &         &   \\
        &   &         & -1  & 0 &   &         &   \\
        &   &         &     &   & 0 &         &   \\
        &   &         &     &   &   & \ddots  &   \\
        &   &         &     &   &   &         & 0
  \end{pmatrix*}
\]
und bringen anschließend die Einsen (und damit auch die Minus-Einsen) durch Vertauschen der Zeilen und Spalten an die richtigen Stellen.
Wir gehen erneut per Induktion über $n$ vor.

\begin{enumerate}
  \item
    Falls $A$ von der Form
    \[
        A
      = \begin{pmatrix}
          *       & \cdots  & *       & 0       \\
          \vdots  & \ddots  & \vdots  & \vdots  \\
          *       & \cdots  & *       & \vdots  \\
          0       & \cdots  & \cdots  & 0
        \end{pmatrix}
    \]
    ist, so lässt sich induktiv vorgehen.
  \item
    Falls $A$ von der Form
    \[
    \begin{pmatrix*}[r]
       0      & 1       & 0       & \cdots  & 0       \\
      -1      & 0       & 0       & \cdots  & 0       \\
       0      & 0       & *       & \cdots  & *       \\
       \vdots & \vdots  & \vdots  & \ddots  & \vdots  \\
       0      & 0       & *       & \cdots  & *
    \end{pmatrix*}
    \]
    ist, so lässt sich Induktion anwenden.
  \item
    Falls die erste Spalte von $A$ (und somit auch die erste Zeile) nur aus Nullen besteht, so vertauschen wir die erste Zeile mit der letzten Zeile, sowie anschließend auch die erste Spalte mit der letzten Spalte.
    Wir erhalten dann eine Matrix der Form
    \[
      \begin{pmatrix}
        *       & \cdots  & *       & 0       \\
        \vdots  & \ddots  & \vdots  & \vdots  \\
        *       & \cdots  & *       & \vdots  \\
        0       & \cdots  & \cdots  & 0
      \end{pmatrix}
    \]
    und sind somit im ersten Fall.
  \item
    Ist $A$ von der Form
    \[
      \begin{pmatrix}
        0       & *       & \cdots  & *       & a_i     & *       & \cdots  & *       \\
        *       & 0       & *       & \cdots  & \cdots  & \cdots  & \cdots  & *       \\
        \vdots  & *       & \ddots  & \ddots  &         &         &         & \vdots  \\
        *       & \vdots  & \ddots  & \ddots  & \ddots  &         &         & \vdots  \\
        -a_i    & \vdots  &         & \ddots  & \ddots  & \ddots  &         & \vdots  \\
        *       & \vdots  &         &         & \ddots  & \ddots  & \ddots  & \vdots  \\
        \vdots  & \vdots  &         &         &         & \ddots  & \ddots  & *       \\
        *       & *       & \cdots  & \cdots  & \cdots  & \cdots  & *       & 0
      \end{pmatrix}
    \]
    so vertauschen wir zunächst die zweite mit der $i$-ten Zeile, sowie anschließend auch die zweite mit der $i$-ten Spalte.
    Wir erhalten hierdurch eine Matrix der Form
    \[
      \begin{pmatrix*}[r]
         0      & a_i     & *       & \cdots  & *       \\
        -a_i    & 0       & *       & \cdots  & *       \\
         *      & *       & *       & \cdots  & *       \\
         \vdots & \vdots  & \vdots  & \ddots  & \vdots  \\
         *      & *       & *       & \cdots  & *
      \end{pmatrix*}.
    \]
    Wir teilen nun die erste Zeile durch $a_i$ sowie anschließend auch die erste Spalte durch $a_i$.
    Hierdurch erhalten wir eine Matrix der Form
    \[
      \begin{pmatrix*}[r]
         0      & 1       & *       & \cdots  & *       \\
        -1      & 0       & *       & \cdots  & *       \\
         *      & *       & *       & \cdots  & *       \\
         \vdots & \vdots  & \vdots  & \ddots  & \vdots  \\
         *      & *       & *       & \cdots  & *
      \end{pmatrix*}.
    \]
    Wir können nun die Einträge $1$ und $-1$ nutzen um mit elementaren Zeilenumformungen alle anderen Einträge aus den ersten beiden Spalten zu eliminieren.
    Da wir nach jeder Zeilenumformung auch die entsprechende Spaltenumformung durchführen, werden in den ersten beiden Zeilen alle anderen Einträge eliminiert.
    Wir erhalten somit eine Matrix der Form
    \[
      \begin{pmatrix*}[r]
         0      & 1       & 0       & \cdots  & 0       \\
        -1      & 0       & 0       & \cdots  & 0       \\
         0      & 0       & *       & \cdots  & *       \\
         \vdots & \vdots  & \vdots  & \ddots  & \vdots  \\
         0      & 0       & *       & \cdots  & *
      \end{pmatrix*}
    \]
    und befinden uns im zweiten Fall.
  \item
    Wir erhalten schließlich eine Matrix der Form
    \[
      \begin{pmatrix*}[r]
         0  & 1 &         &     &   &   &         &   \\
        -1  & 0 &         &     &   &   &         &   \\
            &   & \ddots  &     &   &   &         &   \\
            &   &         &  0  & 1 &   &         &   \\
            &   &         & -1  & 0 &   &         &   \\
            &   &         &     &   & 0 &         &   \\
            &   &         &     &   &   & \ddots  &   \\
            &   &         &     &   &   &         & 0
      \end{pmatrix*}.
    \]
    Durch Vertauschen der Zeilen und Spalten dieser Matrix erhalten wir die gesuchte symplektische Normalform.
\end{enumerate}




\addtocounter{subsection}{-2}
\subsection{}

Wir wenden das obige Verfahren auf die gegebene Matrix an.
Dabei bemerke man, dass der obere linke $(2 \times 2)$-Block bereits in der gewünschten Form ist.
\begingroup
\allowdisplaybreaks
\begin{align*}
  &\,
  \begin{pmatrix*}[r]
     0  &  1  &  2  & 3 \\
    -1  &  0  &  4  & 5 \\
    -2  & -4  &  0  & 6 \\
    -3  & -5  & -6  & 0
  \end{pmatrix*}
  \xlongrightarrow{Z_3 - 2 Z_2}
  \begin{pmatrix*}[r]
     0  &  1  &  2  &  3  \\
    -1  &  0  &  4  &  5  \\
     0  & -4  & -8  & -4  \\
    -3  & -5  & -6  &  0
  \end{pmatrix*}
  \\
  \xlongrightarrow{S_3 - 2 S_2}&\,
  \begin{pmatrix*}[r]
     0  &  1  &  0  &  3  \\
    -1  &  0  &  4  &  5  \\
     0  & -4  &  0  & -4  \\
    -3  & -5  &  4  &  0
  \end{pmatrix*}
  \xlongrightarrow{Z_4 - 3 Z_2}
  \begin{pmatrix*}[r]
     0  &  1  &  0  &   3 \\
    -1  &  0  &  4  &   5 \\
     0  & -4  &  0  &  -4 \\
     0  & -5  & -8  & -15
  \end{pmatrix*}
  \\
  \xlongrightarrow{S_4 - 3 S_2}&\,
  \begin{pmatrix*}[r]
     0  &  1  &  0  & 0 \\
    -1  &  0  &  4  & 5 \\
     0  & -4  &  0  & 8 \\
     0  & -5  & -8  & 0
  \end{pmatrix*}
  \xlongrightarrow{Z_3 + 4 Z_1}
  \begin{pmatrix*}[r]
     0  &  1  &  0  & 0 \\
    -1  &  0  &  4  & 5 \\
     0  &  0  &  0  & 8 \\
     0  & -5  & -8  & 0
  \end{pmatrix*}
  \\
  \xlongrightarrow{S_3 + 4 S_1}&\,
  \begin{pmatrix*}[r]
     0  &  1  &  0  & 0 \\
    -1  &  0  &  0  & 5 \\
     0  &  0  &  0  & 8 \\
     0  & -5  & -8  & 0
  \end{pmatrix*}
  \xlongrightarrow{Z_4 + 5 Z_1}
  \begin{pmatrix*}[r]
     0  & 1 &  0  & 0 \\
    -1  & 0 &  0  & 5 \\
     0  & 0 &  0  & 8 \\
     0  & 0 & -8  & 0
  \end{pmatrix*}
  \\
  \xlongrightarrow{S_4 + 5 S_1}&\,
  \begin{pmatrix*}[r]
     0  & 1 &  0  & 0 \\
    -1  & 0 &  0  & 0 \\
     0  & 0 &  0  & 8 \\
     0  & 0 & -8  & 0
  \end{pmatrix*}
  \xlongrightarrow{\frac{1}{8} Z_3}
  \begin{pmatrix*}[r]
     0  & 1 &  0  & 0 \\
    -1  & 0 &  0  & 0 \\
     0  & 0 &  0  & 1 \\
     0  & 0 & -8  & 0
  \end{pmatrix*}
  \\
  \xlongrightarrow{\frac{1}{8} S_3}&\,
  \begin{pmatrix*}[r]
     0  & 1 &  0  & 0 \\
    -1  & 0 &  0  & 0 \\
     0  & 0 &  0  & 1 \\
     0  & 0 & -1  & 0
  \end{pmatrix*}
  \xlongrightarrow{Z_2 \leftrightarrow Z_3}
  \begin{pmatrix*}[r]
     0  & 1 &  0  & 0 \\
     0  & 0 &  0  & 1 \\
    -1  & 0 &  0  & 0 \\
     0  & 0 & -1  & 0
  \end{pmatrix*}
  \\
  \xlongrightarrow{S_2 \leftrightarrow S_3}&\,
  \begin{pmatrix*}[r]
     0  &  0  & 1 & 0 \\
     0  &  0  & 0 & 1 \\
    -1  &  0  & 0 & 0 \\
     0  & -1  & 0 & 0
  \end{pmatrix*}
  \eqqcolon
  J
\end{align*}
\endgroup
Man bemerke, dass nach jeder simultanen Zeilen- und Spaltenumformung die entstandene Matrix erneut schiefsymmetrisch ist.

Wir bestimme nun eine Matrix $S \in \GL{4}{\Real} $ mit $S^T A S = J$ indem wir die genutzen Spaltenumformungen auf die Einheitsmatrix anwenden:
\begingroup
\allowdisplaybreaks
\begin{align*}
  \begin{pmatrix*}[r]
    1 & 0 & 0 & 0 \\
      & 1 & 0 & 0 \\
      &   & 1 & 0  \\
      &   &   & 1
  \end{pmatrix*}
  &\xlongrightarrow{S_3 - 2 S_2}
  \begin{pmatrix*}[r]
    1 & 0 &  0  & 0 \\
      & 1 & -2  & 0 \\
      &   &  1  & 0  \\
      &   &     & 1
  \end{pmatrix*}
  \xlongrightarrow{S_4 - 3 S_2}
  \begin{pmatrix*}[r]
    1 & 0 &  0  &  0  \\
      & 1 & -2  & -3  \\
      &   &  1  &  0  \\
      &   &     &  1
  \end{pmatrix*}
  \\
  &\xlongrightarrow{S_3 + 4 S_1}
  \begin{pmatrix*}[r]
    1 & 0 &  4  &  0  \\
      & 1 & -2  & -3  \\
      &   &  1  &  0  \\
      &   &     &  1
  \end{pmatrix*}
  \xlongrightarrow{S_4 + 5 S_1}
  \begin{pmatrix*}[r]
    1  & 0 &  4 &  5  \\
       & 1 & -2 & -3  \\
       &   &  1 &  0  \\
       &   &    &  1
  \end{pmatrix*}
  \\
  &\xlongrightarrow{\frac{1}{8} S_3}
  \begin{pmatrix*}[r]
    1 & 0 &  1/2 &  5  \\
      & 1 & -1/4 & -3  \\
      &   &  1/8 &  0  \\
      &   &      &  1
  \end{pmatrix*}
  \xlongrightarrow{S_2 \leftrightarrow S_3}
  \begin{pmatrix*}[r]
    1 &  1/2  & 0 &  5  \\
    0 & -1/4  & 1 & -3  \\
    0 &  1/8  & 0 &  0  \\
    0 &  0    & 0 &  1
  \end{pmatrix*}
  \eqqcolon
  S
\end{align*}
\endgroup




