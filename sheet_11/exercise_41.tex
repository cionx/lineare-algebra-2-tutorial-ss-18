\section{}





\subsection{}

Mit
\[
    S
  = \begin{pmatrix}
      a & b \\
      c & d
    \end{pmatrix}
\]
ergibt sich, dass
\[
    S^T A S
  = \begin{pmatrix}
      - a^2 - c^2 & -ab - cd    \\
      - ab - cd   & - b^2 - d^2
    \end{pmatrix}
\]
gilt.
Es gilt also genau dann $S^T A S = B$ wenn das Gleichungssystem
\[
  \left\{
    \begin{array}{rcr}
      a^2 + c^2 &=&  0 \,,  \\
      b^2 + d^2 &=&  0 \,,  \\
      ab + cd   &=& -1 \,,  \\
    \end{array}
  \right.
\]
erfüllt ist.
Die ersten beiden Gleichungen sind äquivalent zu
\[
  a = \pm i c
  \quad\text{und}\quad
  b = \pm i d \,.
\]
Wir wählen den Ansatz $a = 1$, $c = i$, $b = id$.
Dann wird die dritte Gleichung zu
\[
  2id = -1 \,,
\]
und wir erhalten $d = i/2$ und somit $b = id = -1/2$.
Insgesamt erhalten wir somit, dass sich
\[
    S
  = \begin{pmatrix*}[r]
      1 & -1/2  \\
      i &  i/2
    \end{pmatrix*} \,.
\]
wählen lässt.
Die Invertierbarkeit von $S$ ergibt sich etwa dadurch, dass $\det(S) =  i \neq 0$ gilt, sowie auch dadurch, dass $S^T A S = B$ invertierbar ist.





\subsection{}

Dass die Matrizen $A$ und $B$ über $\Real$ nicht kongruent seien können, lässt sich auf verschiedene Weisen einsehen:

\begin{itemize}
  \item
    Es lässt sich wie im vorherigen Aufgabenteil vorgehen.
    Dann ergibt sich allerdings aus $a^2 + c^2 = b^2 + d^2 = 0$, dass $a = b = c = d = 0$ und somit $S = 0$ gilt.
    Dann gilt aber weder $S^T A S = B$, noch ist $S$ invertierbar.
  \item
    Aus dem Sylvesterschen Trägheitssatz ergibt sich, dass die beiden symmetrischen rellen Matrizen $A, B$ genau dann kongruent zueinander sind wenn sie die gleiche Signatur besitzen.
    Die Signatur von $A$ ist $(0,2,0)$, und aus $\det B = -1 < 0$ ergibt sich, dass die Signatur von $B$ genau $(1,-1,0)$ seien muss.
  \item
    Wären $A$ und $B$ über $\Real$ kongruent, gebe es also ein $S \in \GL{2}{\Real}$ mit $S^T A S = B$, so würde insbesondere
    \[
        \det(B)
      = \det(A) \underbrace{\det(S)^2}_{> 0}
    \]
    gelten.
    Dann müssten $\det A$ und $\det B$ das gleiche Vorzeichen besitzen, was wegen $\det A = 1$ und $\det B = -1$ aber nicht der Fall ist.
\end{itemize}




