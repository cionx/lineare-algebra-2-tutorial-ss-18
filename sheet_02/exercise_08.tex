\section{}





\subsection{}

Aus der Vorlesung ist bekannt, dass Eigenvektoren zu verschiedenen Eigenvektoren linear unabhängig sind, dass also die Summe $\sum_{\lambda \in K} \eigenspace{A}{\lambda}$ direkt ist.
Für jedes $v \in \sum_{\lambda \in K} \eigenspace{A}{\lambda}$ ist also die Darstellung $v = \sum_{\lambda \in K} v_\lambda$ mit
\begin{itemize}
  \item
    $v_{\lambda} \in \eigenspace{A}{\lambda}$ für alle $\lambda \in K$, und
  \item
    $v_\lambda = 0$ für fast alle $\lambda \in K$
\end{itemize}
bereits eindeutig.

Da es sich bei $v$ und $w$ um Eigenvektoren zu verschiedenen Eigenwerten handelt, sind $v$ und $w$ linear unabhängig.
Inbesondere gilt somit $v \neq -w$, und somit $v + w \neq 0$.

Wäre $u \defined v + w$ ein ebenfalls Eigenvektor von $A$, so wären $v + w = u$ zwei Darstellungen des gleichen Vektors $u$ als Summe von Eigenvektoren.
Wie oben erläutert, müssen diese Darstellungen deshalb gleich sein, d.h.\ es gilt entweder $v = u$ und $w = 0$, oder $w = u$ und $v = 0$.
Dies steht aber im Widerspruch dazu, dass $v$ und $w$ Eigenvektoren sind, und somit $v, w \neq 0$ gilt.





\subsection{}

Gilt $A = \lambda E_n$ für ein $\lambda \in K$, so gilt $Av = \lambda E_n v = \lambda v$ für alle $v \in V$, weshalb jedes $v \in V$, $v \neq 0$ ein Eigenvektor von $V$ zum Eigenvektor $\lambda$ ist.

Es sei nun andererseits jeder Vektor $v \in V$, $v \neq 0$ ein Eigenvektor von $V$ zum Eigenvektor $\lambda_v \in V$.

\begin{claim*}
  Für alle $v, w \in V$ mit $v, w \neq 0$ gilt $\lambda_v = \lambda_w$.
\end{claim*}

\begin{proof}
  Gebe es $v, w \in V$, $v, w \neq 0$ mit $\lambda_v \neq \lambda_w$, so wären $v$ und $w$ Eigenvektoren von $V$ zu verschiedenen Eigenwerten.
  Nach dem vorherigen Aufgabenteil würde dann $v + w \neq 0$ gelten, aber $v + w$ kein Eigenvektor von $A$ sein, im Widerspruch zur Annahme.
\end{proof}

Es sind also alle $v \in V$ mit $v \neq 0$ bereits Eigenvektoren zum selben Eigenwert $\lambda \defined \lambda_v \in K$.
Für alle $v \in V$ gilt deshalb $Av = \lambda v = \lambda E_n v$ und somit insgesamt $A = \lambda E_n$.





