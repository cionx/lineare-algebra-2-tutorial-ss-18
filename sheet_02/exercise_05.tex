\section{}

Wir konstruieren iterativ Polynome $q_0, q_1, q_2, \dotsc \in K[X]$ und $r_0, r_1, r_2, \dotsc \in K[X]$ mit den folgenden Eigenschaften:
\begin{itemize}
  \item
    Es gilt $f = q_i g + r_i$ für alle $i \geq 0$.
  \item
    Es gilt $\deg(r_0) > \deg(r_1) > \deg(r_2) > \dotsb$
\end{itemize}
Es gibt dann ein $N \geq 0$, so dass nach $N$ vielen Iterationen $\deg(r_N) < \deg(g)$ gilt, denn wegen $g \neq 0$ gilt $\deg(g) \geq 0$.
Für $g \defined g_N$ und $r \defined r_N$ gilt dann $f = qg + r$ mit $\deg(r) < \deg(g)$.

\begin{enumerate}
  \item
    Wir beginnen mit $r_0 \defined f_0$ und $q_0 \defined 0$.
    Es gilt dann $f = 0 \cdot g + f = q_0 g + r_0$.
  \item
    Gilt $\deg(r_i) < \deg(g)$ für ein $i$, so terminiert der Algorithmus.
  \item
    Gilt $\deg(r_i) \geq \deg(g)$, so werden $q_{i+1}$ und $r_{i+1}$ wie folgt definiert:
    
    Es sei $b X^m$ der Leitterm von $g$, d.h.\ es gelte $g = b X^m + \sum_{k=0}^{m-1} b_k X^k$ mit $b \neq 0$.
    Es sei $a X^n$ der Leitterm von $r_i$, d.h.\ es gelte $r_i = a X^n + \sum_{k=0}^{n-1} a_k X^k$ mit $a \neq 0$.
    Nach Annahme gilt $n \geq m$.
    Wir setzen
    \begin{align*}
      h       \defined \frac{a}{b} X^{n-m} \,,
      \quad
      q_{i+1} \defined q_i + h \,,
      \quad
      r_{i+1} \defined r_i - h g \,.
    \end{align*}
    (Es handelt sich bei $h$ nur um einen Hilsterm, der selber keine Rolle spielt.
    Er dient dazu, die Rechnungen übersichtlicher zu halten.)
    Dann gelten die gewünschten Bedingungen auch für $q_{i+1}$ und $r_{i+1}$:
    \begin{itemize}
      \item
        Es gilt $q_{i+1} g + r_{i+1} = q_i + h g + r_i - h g = q_i g + r_i = f$.
      \item
        Die beiden Polynome $r_i$ und $h g$ haben den gleichen Leitterm $a X^n$, weshalb $r_{i+1} = r_i - h g$ einen echt kleineren Grad hat als $r_i$.
    \end{itemize}
\end{enumerate}


\subsection*{Beispiel: $X^3 + 3X + 3$ durch $X + 1$}

Es seien $f \defined X^3 + 3X + 3$ und $g \defined X + 1$.
Der Leitterm von $g$ ist $X$.

\begin{enumerate}
  \item
    Wir beginnen mit $q_0 \defined 0$ und $r_0 \defined f = X^3 + 3X + 3$.
  \item
    Es gilt $\deg(r_0) = 3 > 1 = \deg(g)$, also setzen wir den Algorithmus fort:
    Der Leitterm von $r_0 = X^3 + 3X + 3$ ist $X^3$.
    Wir setzen also
    \begin{gather*}
                h
      \defined  X^{3-1}
      =         X^2
    \\
                q_1
      \defined  q_0 + h
      =         0 + X^2
      =         X^2
    \shortintertext{und}
                r_1
      \defined  r_0 - h g
      =         (X^3 + 3X + 3) - X^2 (X + 1)
      =         -X^2 + 3X + 3 \,.
    \end{gather*}
  \item
    Es gilt $\deg(r_1) = 2 > 1 = \deg(g)$, also setzen wir den Algorithmus fort:
    Der Letterm von $r_1 = -X^2 + 3X + 3$ ist $-X^2$.
    Wir setzen also
    \begin{gather*}
                h
      \defined  -X^{2-1}
      =         -X
    \\
                q_2
      \defined  q_1 - h
      =         X^2 - X
    \shortintertext{und}
                r_2
      \defined  r_1 - h g
      =         (-X^2 + 3X + 3) - (-X) (X + 1)
      =         4X + 3 \,.
    \end{gather*}
  \item
    Es gilt $\deg(r_2) = 1 = 1 = \deg(g)$, also setzen wir den Algorithmus fort:
    Der Letterm von $r_2 = 4X + 3$ ist $4X$.
    Wir setzen also
    \begin{gather*}
                h
      \defined  4X^{1-1}
      =         4
    \\
                q_3
      \defined  q_2 + h
      =         X^2 - X + 4
    \shortintertext{und}
                r_3
      \defined  r_2 - h g
      =         (4X + 3) - 4 (X + 1)
      =         -1 \,.
    \end{gather*}
  \item
    Es gilt $\deg(r_3) = 0 < 1 = \deg(g)$, also beenden wir den Algorithmus.
\end{enumerate}
Wir erhalten somit die Polynome $q \defined g_3 = X^2 - X + 4$ und $r \defined r_3 = -1$ mit $f = qg + r$ und $\deg(r) < \deg(g)$.

Die obige Rechnung lässt sich schematisch wie folgt zusammenfassen:

\begin{enumerate}
  \item
    \polylongdiv[style=B,stage=1]{X^3 + 3X + 3}{X + 1}
  \item
    \polylongdiv[style=B,stage=2]{X^3 + 3X + 3}{X + 1}
  \item
    \polylongdiv[style=B,stage=3]{X^3 + 3X + 3}{X + 1}
  \item
    \polylongdiv[style=B,stage=4]{X^3 + 3X + 3}{X + 1}
  \item
    \polylongdiv[style=B,stage=5]{X^3 + 3X + 3}{X + 1}
  \item
    \polylongdiv[style=B,stage=6]{X^3 + 3X + 3}{X + 1}
  \item
    \polylongdiv[style=B,stage=7]{X^3 + 3X + 3}{X + 1}
  \item
    \polylongdiv[style=B,stage=8]{X^3 + 3X + 3}{X + 1}
  \item
    \polylongdiv[style=B,stage=9]{X^3 + 3X + 3}{X + 1}
  \item
    \polylongdiv[style=B,stage=10]{X^3 + 3X + 3}{X + 1}
  \item
    \polylongdiv[style=B,stage=11]{X^3 + 3X + 3}{X + 1}
\end{enumerate}


\subsection*{Beispiel: $2X^4 + X^3 + X^2 + X + 1$ durch $2X^2 - 1$}

Schematisch zusammengefasst erhalten wir die folgende Rechnung:
\[
  \polylongdiv[style=B]{2X^4 + X^3 + X^2 + X + 1}{2X^2 - 1}
\]
Für die Polynome $q \defined X^2 + \frac{1}{2} X + 1$ und $r \defined \frac{3}{2} X + 2$ gilt also $f = qg + r$ mit $\deg(r) < \deg(g)$.




