\section{}





\subsection{}

Ist allgemeiner $h \colon V \to V$ ein Endomorphismus und $v \in V$, $v \neq 0$ ein Eigenvektor von $h$ zum Eigenwert $\lambda \in K$, dann ist der Vektor $v$ für jedes Polynom $p \in K[X]$ auch ein Eigenvektor von $p(h)$ zum Eigenwert $p(\lambda)$:
Aus $h(v) = \lambda v$ ergibt sich nämlich induktiv, dass $h^k(v) = \lambda^k v$ für alle $k \geq 0$ gilt;
für $p = \sum_{k=0}^n a_k X^k$ ergibt sich somit, dass
\[
    p(h)(v)
  = \sum_{k=0}^n a_k h^k(v)
  = \sum_{k=0}^n a_k \lambda^k v
  = p(\lambda) v \,.
\]
Gilt insbesondere $p(h) = 0$, so muss jeder Eigenwert $\lambda$ von $h$ eine Nullstelle von $p$ sein. 
Hieraus ergeben sich nun insbesondere die möglichen Eigenwerte von $f$ und $g$:

\begin{example}
  \leavevmode
  \begin{enumerate}
    \item
      Es gilt $f^2 = \id_V$, also $f^2 - \id_V = 0$, und somit $p(f) = 0$ für das Polynom $p \defined X^2 - 1 = (X-1)(X+1)$.
      Also sind $1$ und $-1$ die einzigen möglichen Eigenwerte von $f$.
      Dass beide Werte auch tatsächlich auftreten können, folgt durch Betrachtung des Beispiels $V = K^2$ und
      \[
                  f
        \defined  \begin{pmatrix}
                    1 &     \\
                      & -1
                  \end{pmatrix} \,.
      \]
    \item
      Es gilt $g^2 = cg$, also $g^2 - cg = 0$, und somit $p(g) = 0$ für $p \defined X^2 - cX = X(X-c)$.
      Also sind $0$, $c$ die einzigen möglichen Eigenwerte von $g$.
      Dass beide Werte auch tatsächlich auftreten können, folgt durch Betrachtung des Beispiels $V = K^2$ und
      \[
                  g
        \defined  \begin{pmatrix}
                    0 &   \\
                      & c
                  \end{pmatrix} \,.
      \]
    \item
      Ist $h \colon V \to V$ ein nilpotenter Endomorphismus, so gibt es ein $N \geq 1$ mit $h^N = 0$.
      Für das Polynom $p \defined X^N$ gilt dann $p(h) = 0$.
      Deshalb ist $0$ der einzige mögliche Eigenwert von $h$.
  \end{enumerate}
\end{example}





\subsection*{(ii) + (iii)}

Die hier zu zeigenden Diagonalisierbarkeiten lassen sich wie folgt verallgemeinern:

\begin{lemma}
  \label{lemma: diagonalizable via polynomials}
  Es sei $V$ ein endlichdimensionaler $K$-Vektorraum und $f \colon V \to V$ ein Endomorphismus.
  Es sei $p \in K[X]$ ein Polynom mit $p(f) = 0$, das in paarweise verschiedene Linearfaktoren
  \[
    p = (X - \lambda_1) \dotsm (X - \lambda_n)
  \]
  zerfällt.
  Dann ist $f$ diagonalisierbar mit möglichen Eigenwerten $\lambda_1, \dotsc, \lambda_n$.
\end{lemma}

\begin{proof}
  Es folgt aus $p(f) = 0$, dass $\mu_f \divides p$ gilt.
  Das Minimalpolynom $\mu_f$ ist deshalb von der Form
  \[
      \mu_f
    = (X - \lambda_{i_1}) \dotsm (X - \lambda_{i_m})
  \]
  mit $1 \leq i_1 < i_2 < \dotsb < i_m \leq n$.
  Also zerfällt $\mu_f$ in die paarweise verschiedenen Linearfaktoren $X - \lambda_{i_1}, \dotsc, X - \lambda_{i_m}$.
  Deshalb ist $f$ diagonalisierbar mit Eigenwerten $\lambda_{i_1}, \dotsc, \lambda_{i_m}$.
\end{proof}

\begin{example}
  \label{example: diagonalizable via polynomials}
  \leavevmode
  \begin{enumerate}
    \item
      Es gelte $f^2 = \id_V$, also $p(f) = 0$ für das Polynom
      \[
                  p
        \defined  X^2 - 1
        =         (X - 1)(X + 1)
        \in       K[X] \,.
      \]
      Gilt $1 \neq -1$, also $\ringchar{K} \neq 2$, so ist $f$ also diagonalisierbar mit möglichen Eigenwerten $1, -1$.
    \item
      Es sei allgemeiner $V$ ein endlichdimensionaler $\mathbb{C}$-Vektorraum und $g \colon V \to V$ ein Endomorphismus mit $g^n = \id_V$ für ein $n \geq 1$.
      Dann gilt $p(g) = 0$ für das Polynom $p \defined X^n - 1$.
      Die Nullstellen des Polynoms $p$ sind die $n$-ten Einheitswurzeln, d.h.\ es gilt
      \[
        p = (X - \omega_0) \dotsm (X - \omega_{n-1})
        \quad\text{mit}\quad
        \omega_k \defined e^{2 \pi i k / n} \,.
      \]
      Dabei sind die Einheitswurzeln $\omega_0, \dotsc, \omega_{n-1}$ paarweise verschieden.
      Also ist $g$ diagonalisierbar mit möglichen Eigenwerten $\omega_0, \dotsc, \omega_{n-1}$.
    \item
      Ist $G \subseteq \GL{}{V}$ eine endliche mit $n \defined \card{G}$, so ergibt sich aus grundlegender Gruppentheorie, dass $g^n = 1$ für alle $g \in G$ gilt (wir werden dies hier nicht zeigen).
      Ist also $G \subseteq \GL{}{V}$ eine endliche Untergruppe, so gilt für $n \defined \card{G}$, das $g^n = \id_V$ für alle $g \in G$ gilt.
      Nach dem obigen Beispiel ist deshalb jedes Gruppenelement $g \in G$ diagonalisierbar mit möglichen Eigenwerten $\omega_0, \dotsc, \omega_{n-1}$.
    \item
      Es gelte $g^2 = cg$ für einen Skalar $c \in K$, also $p(g) = 0$ für das Polynom
      \[
                  p
        \defined  X^2 - cX
        =         X(X-c)
        \in       K[X] \,.
      \]
      Gilt $c \neq 0$, also $c \in \unitgroup{K}$, so ist $g$ diagonalisierbar mit möglichen Eigenwerten $0, c$.
    \item
      Das obige Beispiel hat einen wichtigen Sonderfall:
      Es sei $e \colon V \to V$ ein Endomorphismus mit $e^2 = e$.
      Dann gilt $p(e) = 0$ für das Polynom
      \[
                  p
        \defined  X^2 - X
        =         X(X-1) \,.
      \]
      Also ist $e$ diagonalisierbar mit möglichen Eigenwerten $0$, $1$.
      Es gilt somit
      \[
        V = \eigenspace{e}{1} \oplus \eigenspace{e}{0} \,.
      \]
      Zerlegt man $v \in V$ als $v = v_1 + v_0$ mit $v_1 \in \eigenspace{e}{1} \definedas U_1$ und $v_0 \in \eigenspace{e}{0} \eqqcolon U_0$, so gilt
      \[
          e(v)
        = e(v_1 + v_0)
        = e(v_1) + e(v_0)
        = 1 \cdot v_1 + 0 \cdot v_0
        = v_1 \,.
      \]
      Man bezeichnet $e$ deshalb als die \emph{Projektion auf $U_1$ entlang $U_0$}.
      Man bemerke, dass $\im(e) = U_1$ und $\ker(e) = U_0$ gilt.
  \end{enumerate}
\end{example}

\begin{remark}
  Lemma~\ref{lemma: diagonalizable via polynomials} gilt auch falls $V$ unendlichdimensional ist.
  Dann muss allerdings anders argumentiert werden, oder über Minimalpolynome im Unendlichdimensionalen gesprochen werden.
\end{remark}







