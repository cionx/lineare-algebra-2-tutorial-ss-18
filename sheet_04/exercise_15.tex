\section{}

Für alle $\lambda, \mu \in K$ lässt sich
\[
    \eigenspace{ \restrict{g}{\eigenspace{f}{\lambda}} }{\mu}
  = \eigenspace{f}{\lambda} \cap \eigenspace{g}{\mu}
\]
als der \emph{gemeinsame Eigenraum} von $f, g$ zu den Eigenwerten $\lambda, \mu$ bezeichnen;
die Vektoren $v \in \eigenspace{f}{\lambda} \cap \eigenspace{g}{\mu}$ mit $v \neq 0$ sind \emph{gemeinsame Eigenvektoren} von $f$ und $g$.
Wir haben im Tutorium gesehen, dass
\[
    V
  = \bigoplus_{\lambda \in K} \eigenspace{f}{\lambda}
  = \bigoplus_{\lambda \in K} \bigoplus_{\mu \in K} \eigenspace{ \restrict{g}{\eigenspace{f}{\lambda}} }{\mu}
  = \bigoplus_{\lambda, \mu \in K} \bigl[ \eigenspace{f}{\lambda} \cap \eigenspace{g}{\mu} \bigr]
\]
gilt, dass also $V$ in die gemeinsamen Eigenräume von $f$ und $g$ zerfällt.

Ist $h \colon V \to V$ ein weiterer Endomorphismus, der mit $f$ und $g$ kommutiert, so gilt
\[
            h( \eigenspace{f}{\lambda} \cap \eigenspace{g}{\mu}  )
  \subseteq h(\eigenspace{f}{\lambda}) \cap h(\eigenspace{g}{\mu})
  \subseteq \eigenspace{f}{\lambda} \cap \eigenspace{g}{\mu} \,.
\]
Ist $h$ ebenfalls diagonalisierbar, so ergibt sich dann, dass
\begin{align*}
      V
   =  \bigoplus_{\lambda, \mu \in K} \bigl[ \eigenspace{f}{\lambda} \cap \eigenspace{g}{\mu} \bigr]
  &=  \bigoplus_{\lambda, \mu \in K}
      \bigoplus_{\kappa \in K}
      \eigenspace{ \restrict{h}{ \eigenspace{f}{\lambda} \cap \eigenspace{g}{\mu} } }{\kappa} \\
  &=  \bigoplus_{\lambda, \mu, \kappa \in K}
      \bigl[ \eigenspace{f}{\lambda} \cap \eigenspace{g}{\mu} \cap \eigenspace{h}{\kappa} \bigr] \,.
\end{align*}
Induktiv lässt sich damit die folgende Aussage zeigen:

\begin{proposition}
  \label{proposition: equivalent conditions for simultaneously diagonalizable}
  Für Endomorphismus $f_1, \dotsc, f_n \colon V \to V$ sind die folgenden Bedingungen äquivalent:
  \begin{enumerate}
    \item
      Es gibt eine Basis von $V$, die aus gemeinsamen Eigenvektoren von $f_1, \dotsc, f_n$ besteht.
    \item
      Die Endomorphismen $f_1, \dotsc, f_n$ sind diagonalisierbar und paarweise kommutierend.
    \item
      Es gilt die Zerlegung
      \[
          V
        = \bigoplus_{\lambda_1, \dotsc, \lambda_n \in K}
          \bigl[ \eigenspace{f_1}{\lambda_1} \cap \dotsb \cap \eigenspace{f_n}{\lambda_n} \bigr]
      \]
      in die gemeinsame Eigenräume der Endomorphismen $f_1, \dotsc, f_n$.
  \end{enumerate}
\end{proposition}

Man bezeichnet Endomorphismen $f_1, \dotsc, f_n \colon V \to V$, die eine (und damit alle) der Bedinungen aus Proposition~\ref{proposition: equivalent conditions for simultaneously diagonalizable} erfüllen, als \emph{simultan diagonalisierbar}.
Simultane Diagonalisierbarkeit spielt eine wichtige Rolle in der Darstellungstheorie, sowie in der Quantenmechanik.

\begin{example}
  Es sei $V$ ein endlichdimensionaler $\mathbb{C}$-Vektorraum und $G \subseteq \GL{V}$ eine endliche Untergruppe.
  Wir haben in Beispiel~\ref{example: diagonalizable via polynomials} gesehen, dass für $n \defined \card{G}$ jedes $g \in G$ diagonalisierbar mit möglichen Eigenwerten $\omega_0, \dotsc, \omega_{n-1}$ ist.
  
  Ist $G$ abelsch, so folgt aus der obigen Argumentation, dass $G$ simultan diagonalisierbar ist.
  Es gibt dann eine Basis $B$ von $V$, so dass $c_{B,B}(g)$ für jedes $g \in G$ eine Diagonalmatrix mit möglichen Diagonaleinträgen $\omega_0, \dotsc, \omega_{n-1}$ ist.
  Hieraus ergeben sich beispielsweise die folgenden beiden Resultate:
  \begin{enumerate}
    \item
      Jede endliche abelsche Untergruppe von $\GL{V}$ ist isomorph zu einer Untergruppe von $\diagonalgroup{n}{\Complex}$ (die Gruppe der invertieren $(n \times n)$-Diagonalmatrizen über $\Complex$).
    \item
      Der Vektorraum $V$ lässt sich als $V = V_1 \oplus \dotsb \oplus V_d$ zerlegen, wobei $V_i \subseteq V$ für jedes $i = 1, \dotsc, d$ ein eindimensionaler Untervektorraum ist, der $g$-invariant für jedes $g \in G$ ist (also $G$-invariant).
      (Gilt $B = (b_1, \dotsc, b_d)$, so wähle man $V_i = \Lin(b_i)$.)
      
      In der Sprache der Darstellungstheorie endlicher Gruppen bedeutet dies, dass die Darstellung $V$ von $G$ in eindimensionale Unterdarstellungen zerfällt, bzw.\ dass jede irreduzible komplexe Darstellung einer endlichen abelschen Gruppe bereits eindimensional ist.
  \end{enumerate}
\end{example}
