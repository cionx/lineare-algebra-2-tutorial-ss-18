\section{}

Diese Aufgabe lässt sich wie folgt verallgemeinern:
Es sei $(a_n)_{n \geq 0}$ eine Folge, die durch Startwerte $a_0, \dotsc, a_{d-1} \in K$ und die Rekursionsvorschrift
\[
    a_{n+d}
  = c_{d-1} a_{n+d-1} + \dotsb + c_1 a_{n+1} + c_0 a_{n}
\]
gegeben wird.
Für die Matrix
\[
            A
  \coloneqq \begin{pmatrix}
              c_{d-1} & \cdots  & c_1 & c_0     \\
              1       &         &     & 0       \\
                      & \ddots  &     & \vdots  \\
                      &         & 1   & 0
            \end{pmatrix}
  \in       \matrices{d}{K}
\]
gilt dann
\[
    A \columnvector{a_{n+d-1} \\ \vdots \\ a_n}
  =   \columnvector{a_{n+d} \\ \vdots \\ a_{n+1}}
\]
für alle $n \geq 0$.
Das charakteristische Polynom der Matrix $A$ ist durch
\[
              \charpoly{A}
  =           X^d - c_{d-1} X^{d-1} - \dotsb - c_1 X - c_0
  \definedas  p
\]
gegeben.
Sofern sich $A$ diagonalisieren lässt (falls etwa $p$ in paarweise verschiedene Linearfaktoren zerfällt), lässt sich nun eine explizite Formel für die Folgenglieder $a_n$ bestimmen.

\begin{remark}
  Die Matrix $A$ lässt sich als eine Art von Begleitmatrix für das Polynom $p$ auffasen.
  Anstelle von $A$ könnte man auch die Matrix
  \[
              A'
    \coloneqq \begin{pmatrix}
                0       & 1       &         &         \\
                \vdots  &         & \ddots  &         \\
                0       &         &         & 1       \\
                c_0     & c_{d-2} & \cdots  & c_{d-1}
              \end{pmatrix}
  \]
  betrachten, für die
  \[
      A' \columnvector{a_n \\ \vdots \\ a_{n+d-1}}
    =    \columnvector{a_{n+1} \\ \vdots \\ a_{n+d}}
  \]
  gilt.
  Dann ist $A'$ die Transponierte der Begleitmatrix von $p$, wodurch sich sofort ergibt, dass $\chi_{A'} = p$ gilt.
  In dieser Aufgaben wurde allerdings die Konvention für $A$ genutzt, und nicht die Konvention für $A'$.
\end{remark}


