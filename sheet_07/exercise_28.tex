\section{}

Es gilt
\[
    M_X(A)
  = \begin{pmatrix}
       X  & -1  & -1  \\
      -1  &  X  & -1  \\
      -1  & -1  &  X
    \end{pmatrix}.
\]
Durch Vertauschen der ersten beiden Zeilen erhalten wir die Matrix
\[
  \begin{pmatrix}
    -1  &  X  & -1  \\
     X  & -1  & -1  \\
    -1  & -1  &  X
  \end{pmatrix}.
\]
Durch Multiplikation der ersten Zeile mit $-1$ erhalten wir die Matrix
\[
  \begin{pmatrix}
     1  & -X  &  1  \\
     X  & -1  & -1  \\
    -1  & -1  &  X
  \end{pmatrix}.
\]
Durch Subtrakten des $X$-fachen der ersten Zeile von der zweiten Zeile, sowie Addition der ersten Zeile auf die zweite Zeile erhalten wir die Matrix
\[
  \begin{pmatrix}
    1 &    -X   &    1  \\
    0 & X^2  -1 & -X-1  \\
    0 &    -X-1 &  X+1
  \end{pmatrix}.
\]
Durch Addition des $X$-fachen der ersten Spalte auf die zweite Spalte, sowie Subtraktion der ersten Spalte von der dritten Spalte erhalten wir die Matrix
\[
  \begin{pmatrix}
    1 &       0 &    0  \\
    0 & X^2  -1 & -X-1  \\
    0 &    -X-1 &  X+1
  \end{pmatrix}.
\]
Durch Vertauschen der zweiten und dritten Zeile, sowie Vertauschen der zweiten und dritten Spalte erhalten wir die Matrix
\[
  \begin{pmatrix}
    1 &    0  &        0  \\
    0 &  X+1  &     -X-1  \\
    0 & -X-1  &  X^2  -1
  \end{pmatrix}.
\]
Durch Addition der zweiten Zeile auf die dritte Zeile erhalten wir die Matrix
\[
  \begin{pmatrix}
    1 &    0  &        0  \\
    0 &  X+1  &     -X-1  \\
    0 &    0  &  X^2-X-2
  \end{pmatrix}.
\]
Durch Addition der zweiten Spalte auf die dritte Spalte erhalten wir die Matrix
\[
  \begin{pmatrix}
    1 &    0  &        0  \\
    0 &  X+1  &        0  \\
    0 &    0  &  X^2-X-2
  \end{pmatrix}.
\]
Es gilt $X^2 - X - 2 = (X + 1)(X - 2)$ und somit $(X + 1) \mid (X^2 - X - 2)$.
Die letzte Matrix befindet sich also in Smith-Normalform.

Man beachte, dass die obige Rechnung über jedem Körper gültig ist.
Für $K = \Finite_2$ gilt dabei $X^2 - X - 2 = X^2 + X$, und für $K = \Finite_3$ gilt $X^2 - X - 2 = X^2 + 2X + 1 = (X+1)^2$.

\begin{remark}
  Die Existenz und Eindeutigkeit der Smith-Normalform gilt für beliebige $(m \times n)$-Matrizen mit Einträgen in einem Ring $R$, in dem ein \enquote{Teilen mit Rest} möglich ist (sogenannte \emph{euklidische Ringe}\,\footnote{
  Ein euklidischer Ring ist ein Integritätsbereich $R$ zusammen mit einer \emph{Gradabbildung} $\delta \colon R \smallsetminus \{0\} \to \Natural$, so dass es für alle $f, g \in R$, $g \neq 0$ Elemente $q , r \in R$ mit $f = qg + r$ gibt, so dass entweder $r = 0$ oder $\delta(r) < \delta(g)$ gilt.
  Beispiele sind $R = K[X]$ für einen Körper $K$ mit $\delta = \deg$, sowie $R = \Integer$ mit $\delta = |\cdot|$.}).
  Neben $R = K[X]$ gilt dies also insbesondere für $R = \Integer$.
  Der in der Vorlesung angegebene Algorithmus zur Berechnung der Smith-Nomalform funktioniert dabei auch weiterhin.
  (Anstelle von $\unitgroup{K[X]} = \unitgroup{K}$ muss $\unitgroup{\Integer} = \{1, -1\}$ genutzt werden.)
  
  Das Berechnen der Smith-Normalform lässt sich deshalb auch mit $\Integer$-wertigen Matrizen üben.
  Dies hat den Vorteil, dass das Teilen mit Rest in $\Integer$ häufig einfacher durchzuführen ist als in $K[X]$.
\end{remark}

