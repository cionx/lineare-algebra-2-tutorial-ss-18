\section{}

Es gilt
Die Smith-Normalform von $A$ ergibt sich durch
\begin{align*}
    M_X(A)
  =&\,
    \begin{pmatrix}
      X  & -1  & -1  \\
      -1  &  X  & -1  \\
      -1  & -1  &  X
    \end{pmatrix}
  \xrightarrow{Z_1 \leftrightarrow Z_2}\,
    \begin{pmatrix}
      -1  &  X  & -1  \\
       X  & -1  & -1  \\
      -1  & -1  &  X
    \end{pmatrix}
  \\
  \xrightarrow{Z_1 \to -Z_1}&\,
    \begin{pmatrix}
       1  & -X  &  1  \\
       X  & -1  & -1  \\
      -1  & -1  &  X
    \end{pmatrix}
  \xrightarrow{\substack{Z_2 \to Z_2 - X Z_1 \\ Z_3 \to Z_3 + Z_1}}\,
    \begin{pmatrix}
      1 &    -X   &    1  \\
      0 & X^2  -1 & -X-1  \\
      0 &    -X-1 &  X+1
    \end{pmatrix}
  \\
  \xrightarrow{\substack{S_2 \to S_2 + X S_1 \\ S_3 \to S_3 - S_1}}&\,
    \begin{pmatrix}
      1 &       0 &    0  \\
      0 & X^2  -1 & -X-1  \\
      0 &    -X-1 &  X+1
    \end{pmatrix}
  \xrightarrow{Z_2 \leftrightarrow Z_3}\,
    \begin{pmatrix}
      1 &    0  &        0  \\
      0 &  X+1  &     -X-1  \\
      0 & -X-1  &  X^2  -1
    \end{pmatrix}
  \\
  \xrightarrow{Z_3 \to Z_3 + Z_2}&\,
    \begin{pmatrix}
      1 &    0  &        0  \\
      0 &  X+1  &     -X-1  \\
      0 &    0  &  X^2-X-2
    \end{pmatrix}
  \xrightarrow{S_3 \to S_3 + S_2}\,
    \begin{pmatrix}
      1 &    0  &        0  \\
      0 &  X+1  &        0  \\
      0 &    0  &  X^2-X-2
    \end{pmatrix}
  \\
  =&\,
    \begin{pmatrix}
      1 &    0  &              0  \\
      0 &  X+1  &              0  \\
      0 &    0  &  (X + 1)(X - 2)
    \end{pmatrix}
\end{align*}
Die letzte Matrix befindet sich in Smith-Normalform, da $(X + 1) \divides (X + 1)(X - 2)$ gilt.
Man beachte, dass die obige Rechnung über jedem Körper gültig ist;
für $K = \Finite_2$ gilt dabei $X^2 - X - 2 = X^2 + X$, und für $K = \Finite_3$ gilt $X^2 - X - 2 = X^2 + 2X + 1 = (X+1)^2$.

\begin{remark}
  Die Existenz und Eindeutigkeit der Smith-Normalform gilt für beliebige $(m \times n)$-Matrizen mit Einträgen in einem Ring $R$, in dem ein \enquote{Teilen mit Rest} möglich ist (einen sogenannten \emph{euklidischen Ring}\,\footnote{
  Ein euklidischer Ring ist ein Integritätsbereich $R$ zusammen mit einer \emph{Gradabbildung} $\delta \colon R \smallsetminus \{0\} \to \Natural$, so dass es für alle $f, g \in R$, $g \neq 0$ Elemente $q , r \in R$ mit $f = qg + r$ gibt, so dass entweder $r = 0$ oder $\delta(r) < \delta(g)$ gilt.
  Beispiele sind $R = K[X]$ für einen Körper $K$ mit $\delta = \deg$, sowie $R = \Integer$ mit $\delta = |\cdot|$.}).
  Neben $R = K[X]$ gilt dies also insbesondere für $R = \Integer$:
  
  \begin{theorem}
    Es sei $A \in \mnatrices{m}{n}{\Integer}$.
    Dann lässt sich die Matrix $A$ durch elementare Zeilen- und Spaltenumformungen in die Form
    \[
      \begin{pmatrix}
        c_1     &         &         & 0       & \cdots  & 0       \\
                & \ddots  &         & \vdots  &         & \vdots  \\
                &         & c_t     & \vdots  &         & \vdots  \\
        0       & \cdots  & \cdots  & 0       &         & \vdots  \\
        \vdots  &         &         &         & \ddots  & \vdots  \\
        0       & \cdots  & \cdots  & \cdots  & \cdots  & 0
      \end{pmatrix}
    \]
    bringen, so dass $c_1, \dotsc, c_t \neq 0$ gilt, und $c_i \divides c_{i+1}$ für alle $1 \leq i < t$ gilt.
    Die Einträge $c_1, \dotsc, c_t$ sind dabei jeweils bis auf Vorzeichen eindeutig bestimmt.
  \end{theorem}
  
  Der in der Vorlesung angegebene Algorithmus zur Berechnung der Smith-Nomalform funktioniert dabei ebenfalls für $R = \Integer$.
  (Anstelle von $\unitgroup{K[X]} = \unitgroup{K}$ muss dabei $\unitgroup{\Integer} = \{1, -1\}$ genutzt werden.
  Als elementare Zeilen- und Spaltenumformungen vom Typ~(II) darf also nur mit den Skalaren $\pm 1$ multipliziert werden.)
  
  Das Berechnen der Smith-Normalform lässt sich deshalb auch mit $\Integer$-wertigen Matrizen üben.
  Dies hat den Vorteil, dass das Teilen mit Rest in $\Integer$ im Allgemeinen einfacher durchzuführen ist als in $K[X]$.
\end{remark}

