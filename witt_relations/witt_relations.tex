\documentclass[a4paper,10pt,numbers = noenddot]{scrartcl}
\usepackage{../generalstyle}

%opening
\subject{Bemerkungen und Lösungen zu}
\title{Lineare Algebra II}
\subtitle{Beweis von Satz~19.10}
\author{Jendrik Stelzner}
\date{\today}


\begin{document}

\maketitle


Es seien $[a_1, \dotsc, a_n], [b_1, \dotsc, b_n] \in \GL{n}{K}$ zwei invertierbare Diagonalmatrizen mit
\[
          [a_1, \dotsc, a_n]
  \equiv  [b_1, \dotsc, b_n] \,.
\]
Wir erklären im Folgenden, wie sich aus der Matrix $[a_1, \dotsc, a_n]$ durch Umformungen vom Typ~(1) und Typ~(2) eine invertierbare Diagonalmatrix $[\tilde{a}_1, \dotsc, \tilde{a}_n] \in \GL{n}{K}$ ergibt, welche die folgenden beiden Bedingungen erfüllt:
\begin{enumerate}[label = \alph*)]
  \item
    \label{enumerate: b1 is sum}
    Es gibt ein $1 \leq k \leq n$ mit $b_1 = \tilde{a}_1 + \dotsb + \tilde{a}_k$,
  \item
    \label{enumerate: leading sums are nonzero}
    und es gilt $\tilde{a}_1 + \dotsb + \tilde{a}_i \neq 0$ für alle $i = 1, \dotsc, k$.
\end{enumerate}

Hierfür überführen wir die Matrix $[a_1, \dotsc, a_n]$ schrittweise (mit Umformungen vom Typ~(1) und Typ~(2)) in die gewünschte Form $[\tilde{a}_1, \dotsc, \tilde{a}_n]$.

\begin{enumerate}[label = \arabic*)]
  \item
    Da $[a_1, \dotsc, a_n] \equiv [b_1, \dotsc, b_n]$ gilt, gibt es eine Matrix $S = (s_{ij})_{ij} \in \GL{n}{K}$ mit
    \[
      [b_1, \dotsc, b_n] = S^T [a_1, \dotsc, a_n] S \,.
    \]
    Durch Vergleichen des ersten Diagonaleintrags beider Matrizen ergibt sich, dass
    \begin{equation}
      \label{equation: first sum}
      b_1 = \sum_{i=1}^n s_{i1}^2 a_i
    \end{equation}
    gilt.
    Wir betrachten die Diagonalmatrix $[a'_1, \dotsc, a'_n]$ mit
    \[
        a'_{i}
      = \begin{cases}
          s_{i1}^2 a_i  & \text{falls $s_{i1} \neq 0$}, \\
          a_i           & \text{sonst},                 \\
        \end{cases}
    \]
    für alle $i = 1, \dotsc, n$.
    Es gilt $a_i \neq 0$ für alle $i = 1, \dotsc, n$ da $[a_1, \dotsc, a_n]$ invertierbar ist.
    Deshalb gilt auch $a'_i \neq 0$ für alle $i = 1, \dotsc, n$, weshalb auch $[a'_1, \dotsc, a'_n]$ invertierbar ist.
    Die Diagonalmatrix $[a'_1, \dotsc, a'_n]$ geht aus $[a_1, \dotsc, a_n]$ durch wiederholte Umformungen vom Typ~(2) hervor.
    
  \item
    Für die Indexmenge
    \[
                I
      \defined  \{ i \in \{1, \dotsc, n\} \suchthat s_{i1} \neq 0 \}
    \]
    erhalten wir aus der Gleichung~\eqref{equation: first sum}, dass 
    \[
        b_1
      = \sum_{i=1}^n s_{i1}^2 a_i
      = \sum_{i \in I} s_{i1}^2 a_i
      = \sum_{i \in I} a'_i \,.
    \]
    
    Indem wir die Diagonaleinträge der Matrix $[a'_1, \dotsc, a'_n]$ mithilfe einer Umformung vom Typ~(1) passend umsortieren, erhalten wir somit eine invertierbare Diagonalmatrix $[a''_1, \dotsc, a''_n] \in \GL{n}{K}$, so dass für $k'' \defined |I|$ die Gleichung
    \[
        b_1
      = \sum_{i \in I} a'_i
      = \sum_{i=1}^{k''} a''_i
    \]
    gilt.
    
  \item
    Wenn $[a''_1, \dotsc, a''_n]$ mit $k''$ wie oben die Bedingung~\ref*{enumerate: leading sums are nonzero} noch nicht erfüllt, dann gibt es ein $1 \leq i \leq k''$ mit 
    \[
        a''_1 + \dotsb + a''_i
      = 0 \,.
    \]
    Wir sortieren dann die Diagonaleinträge von $[a''_1, \dotsc, a''_n]$ mithilfe einer Umformung vom Typ~(1), so um, dass wir die invertierbare Diagonalmatrix
    \[
        [a'''_1, \dotsc, a'''_n]
      = [a''_{i+1}, \dotsc, a''_{k''}, a''_1, \dotsc, a''_i, a''_{k''+1}, \dotsc, a''_n] \,.
    \]
    erhalten.
    Es gilt nun, dass
    \[
        b_1
      = \sum_{j=1}^{k''} a''_j
      = \underbrace{ \sum_{j=1}^i a''_j }_{=0} + \sum_{j=i+1}^{k''} a''_j
      = \sum_{j=i+1}^{k''} a''_j
      = \sum_{j=1}^{k''-i} a'''_j \,.
    \]
    Für $k''' \defined k''- i$ erfüllt also die Matrix $[a'''_1, \dotsc, a'''_n]$ auch weiterhin die Bedingung~\ref*{enumerate: b1 is sum}.
    Dabei gilt nun allerdings $k''' < k''$.
  \item
    Indem wir den obigen Schritt endlich oft wiederholen, erhalten wir schließlich eine invertierbare Diagonalmatrix $[\tilde{a}_1, \dotsc, \tilde{a}_n] \in \GL{n}{K}$ und ein $k = 0, \dotsc, n$, so dass die Bedingungen \ref*{enumerate: b1 is sum} und \ref*{enumerate: leading sums are nonzero} beide erfüllt sind.
    
    Die Matrix $[\tilde{a}_1, \dotsc, \tilde{a}_n]$ geht dabei durch endliche viele Umformungen vom Typ~(1) aus der Matrix $[a''_1, \dotsc, a''_n]$ hervor.
    Ingesamt geht die Diagonalmatrix $[\tilde{a}_1, \dotsc, \tilde{a}_n]$ deshalb durch Umformungen vom Typ~(1) und Typ~(2) aus der ursprünglichen Diagonalmatrix $[a_1, \dotsc, a_n]$ hervor.
    
    Wir bemerken schließlich noch, dass $k \geq 1$ gelten muss, denn sonst würde
    \[
        b_1
      = \sum_{i=1}^k \tilde{a}_i = 0
    \]
    gelten, was im Widerspruch zur Invertierbarkeit von $[b_1, \dotsc, b_n]$ stünde.
\end{enumerate}


\end{document}
