\section{}





\subsection{}

Wir konstruieren explizit einen gemeinsamen Teiler von $p$ und $\chi_f$:

Würde bereits $p(f)(u) = 0$ für alle $u \in V$ gelten, so würde $\minpol{f} \divides p$ gelten.
Da nach dem Satz von Cayley-Hamilton auch $\minpol{f} \divides \charpol{f}$ gilt, wäre dann $\minpol{f}$ ein gemeinsamer Teiler von $p$ und $\chi_f$.
Dabei gilt $\deg(\minpol{f}) \geq 1$, da $V \neq 0$ gilt.
Damit wäre dann $\minpol{f}$ der gewünschte gemeinsame Teiler von $p$ und $\charpol{f}$.

Das Problem, dass nicht notwendigerweise $p(f)(u) = 0$ für alle $u \in V$ gilt, lösen wir im Folgenden dadurch, dass wir $V$ durch einen passenden $f$-invarianten Untervektorraum $U \subseteq V$ ersetzten:

Es sei
\[
            U
  \defined  \Lin(v, f(v), f^2(v), \dotsc)
  =         \Lin( \{ f^k(v) \suchthat k \geq 0 \} )
\]
der von $v$ erzeugte $f$-invariante Untervektorraum;
der Untervektorraum $U \subseteq V$ ist $f$-invariant, da $f(f^k(v)) = f^{k+1}(v) \in U$ für alle $k \geq 0$ gilt.
Für alle $k \geq 0$ gilt
\[
    p(f)( f^k(v) )
  = f^k( p(f)(v) )
  = f^k( 0 )
  = 0 \,,
\]
und deshalb gilt $\restrict{p(f)}{U} = 0$.
Es gilt deshalb $\minpol{\restrict{f}{U}} \divides p$.
Andererseits gilt auch
\[
    \charpol{f}(\restrict{f}{U})
  = \restrict{\charpol{f}(f)}{U}
  = \restrict{0}{U}
  = 0
\]
und somit $\minpol{\restrict{f}{U}} \divides \charpol{f}$.
Also ist $\minpol{\restrict{f}{U}}$ ein gemeinsamer Teiler von $p$ und $\charpol{f}$.
Wegen $v \neq 0$ gilt dabei $U \neq 0$ und somit $\deg( \minpol{\restrict{f}{U}} ) \geq 1$.
Also ist $\minpol{\restrict{f}{U}}$ der gesuchte gemeinsame Teiler von $p$ und $\charpol{f}$.





\subsection{}

Wir geben noch eine alternative Lösung für diesen Aufgabenteil an.
Entscheidend ist hierfür die folgende Beobachtung:

\begin{lemma}
  Ist $U \subseteq V$ ein $f$-invarianter Untervektorraum, so gilt $\charpol{\restrict{f}{U}} \divides \charpol{f}$.
\end{lemma}

\begin{proof}
  Es sei $B' = (b_1, \dotsc, b_m)$ eine Basis von $U$.
  Durch Basisergänzung ergibt sich eine Basis $B = (b_1, \dotsc, b_m, b_{m+1}, \dotsc, b_n)$ von $V$.
  Dann gilt
  \[
              A
    \defined  \repmatrixendo{f}{B}
    =         \begin{pmatrix}
                A_1 & *   \\
                    & A_2
              \end{pmatrix}
    \in       \matrices{n}{K}
  \]
  mit $A_1 = \repmatrixendo{\restrict{f}{U}}{B'} \in \matrices{m}{K}$ und $A_2 \in \matrices{n-m}{K}$.
  Es gilt somit, dass
  \[
              \charpol{\restrict{f}{U}}
    =         \det(X E_m - A_1)
    \divides  \det(X E_m - A_1) \det(X E_{n-m} A_2)
    =         \det(X E_n - A)
    =         \charpol{f} \,.
  \]
\end{proof}

Für jeden $m$-dimensionalen $f$-invarianten Untervektorraum $U \subseteq V$ ist also $\charpol{\restrict{f}{U}}$ ein Teiler des irreduziblen Polynoms $\charpol{f}$.
Also gilt entweder $\charpol{\restrict{f}{U}} = 1$, und somit $m = \deg(\charpol{\restrict{f}{U}}) = 0$ und deshalb $U = 0$, oder $\charpol{\restrict{f}{U}} = \charpol{f}$, und somit $m = \deg(\charpol{\restrict{f}{U}}) = n$ und deshalb $U = V$.
Das zeigt, dass $0$ und $V$ die einzigen $f$-invarianten Untervektorräume von $V$ sind.

Es sei nun $U \defined \Lin(v, f(v), \dotsc, f^{n-1}(v))$.
Dann ist $U$ ein $f$-invarianter Untervektorraum von $V$:
Für alle $1 \leq k < n-1$ gilt $f(f^k(v)) = f^{k+1}(v) \in U$, und nach dem Satz von Cayley-Hamilton gilt für $\chi_f = X^n + \sum_{k=0}^{n-1} a_k X^k$, dass
\[
    0
  = \charpol{f}(f)(v)
  = f^n(v) + a_{n-1} f^{n-1}(v) + \dotsb + a_1 f(v) + a_0 v \,,
\]
und deshalb auch
\[
    f(f^{n-1}(v))
  = f^n(v)
  = - a_{n-1} f^{n-1}(v) - a_{n-2} f^{n-2}(v) - \dotsb - a_1 f(v) - a_0 v
  \in U \,.
\]
Wir erhalten somit, dass $U = 0$ oder $U = V$ gilt.
Der Fall $U = 0$ kann nicht eintreten, da $v \in U$ mit $v \neq 0$ gilt, also gilt $U = V$.
Deshalb ist die $n$-elementige Familie $(v, \dotsc, f^{n-1}(v))$ ein Erzeugendensystem des $n$-dimensionalen Vektorraums $V$, und es muss sich bereits um eine Basis handeln.




