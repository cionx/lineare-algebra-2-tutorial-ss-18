\documentclass[a4paper,10pt,numbers = noenddot]{scrartcl}
\usepackage{../generalstyle}

%opening
\subject{Bemerkungen und Lösungen zu}
\title{Lineare Algebra II}
\subtitle{Bestimmung der Jordan-Normalform}
\author{Jendrik Stelzner}
\date{Zuletzt bearbeitet: \today}



\begin{document}

\maketitle





\section*{Kochrezept für die Jordan-Normalform}

Ist $A \in \matrices{n}{K}$ eine Matrix, deren charakteristisches Polynom $\charpol{A}$ in Linearfaktoren zerfällt, so lässt sich die Jordan-Normalform von $A$ sowie anschließend auch eine zugehörige Jordanbasis nach dem folgenden \emph{Kochrezept} berechnen:

\begin{itemize}
  \item
    Man bestimme die Eigenwerte von $A$, etwa indem man das charakteristische Polynom $\charpol{A}$ berechnet, und anschließend die Nullstellen von $\charpol{A}$ herausfindet.
    
  \item
    Für jeden Eigenwert $\lambda$ von $A$ führe man die folgenden Schritte durch:
    \begin{itemize}
      \item
        Man berechne die verallgemeinerten Eigenräume
        \[
          \ker (A - \lambda E_n)\,,\,
          \ker (A - \lambda E_n)^2\,,\,
          \dotsc\,,\,
          \ker (A - \lambda E_n)^m
        \]
        bis zu dem Punkt, an dem eine der folgenden äquivalenten Bedingungen erfüllt sind:
        \begin{itemize}
          \item
            Die Dimension von $\ker (A - \lambda E_n)^m$ ist gleich der algebraische Vielfachheit des Eigenwerts $\lambda$ in $\charpol{A}$.
          \item
            Es gilt $\ker (A - \lambda E_n)^m = \ker (A - \lambda E_n)^{m+1}$.
        \end{itemize}
      \item
        Man bestimme Anhand der Zahlen $d_k(\lambda) \defined \dim \ker (A - \lambda E_n)^k$ die Anzahl der auftretenden Jordanblöcke zu $\lambda$ von Größe $k$ als
        \[
                    b_k(\lambda)
          \defined  2 d_k(\lambda) - d_{k-1}(\lambda) - d_{k+1}(\lambda) \,.
        \]
    \end{itemize}
\end{itemize}

Aus den Eigenwerten $\lambda_1, \lambda_2, \dotsc$ von $A$ und den Zahlen $b_k(\lambda_i)$ erhalten wir bereits, wieviele Jordanblöcke von welcher Größe zu welchen Eigenwerten in der Jordan-Normalform auftauchen d.h.\ wie die Jordan-Normalform von $A$ (bis auf Permutation der Blöcke) aussehen wird.
Inbesondere ist $d_1(\lambda) = \dim \ker (A - \lambda E_n)$ die Gesamtzahl der Jordanblöcke zu $\lambda$ und die entsprechende Potenz $m$ die maximale auftretende Blöckgröße zu $\lambda$.

Zur Berechnung einer Jordanbasis für $A$ kann man weiterhin wie folgt vorgehen:

\begin{itemize}[resume]
  \item
    Für jeden Eigenwert $\lambda$ von $A$ berechnet man eine Basis $B_\lambda$ von $\hauptspace{A}{\lambda}$ wie folgt:
    \begin{itemize}
      \item
        Man wähle Vektoren $v_1, \dotsc, v_{b_m} \in \ker (A - \lambda E_n)^m$ mit
        \[
                    \ker (A - \lambda E_n)^m
          =         \ker (A - \lambda E_n)^{m-1}
            \oplus  \Lin(v_1, \dotsc, v_{b_m}) \,.
        \]
        (Zur konkreten Berechnung ergänze man eine Basis $\ker (A - \lambda E_n)^{m-1}$ zu einer Basis von $\ker (A - \lambda E_n)^m$; dann kann man $v_1, \dotsc, v_{b_m}$ als die neu hinzugekommenen Basisvektoren wählen.)
      \item
        Hierdurch ergeben sich für $B_\lambda$ die ersten paar Basisvektoren
        \begin{align*}
          (A - \lambda E_n)^{m-1} v_1\,,\,      &\dotsc,\,  (A - \lambda E_n) v_1\,,\,      v_1\,,      \\
          (A - \lambda E_n)^{m-1} v_2\,,\,      &\dotsc,\,  (A - \lambda E_n) v_2\,,\,      v_2\,,      \\
                                                &\dotsc\,,                                              \\
          (A - \lambda E_n)^{m-1} v_{b_m}\,,\,  &\dotsc,\,  (A - \lambda E_n) v_{b_m}\,,\,  v_{b_m}\,.
        \end{align*}
        Dabei bilden die Basisvektoren aus derselben Zeile jeweils eine Jordankette zum Eigenwert $\lambda$ von Länge $m$.
      \item
        Man wählt nun Vektoren $v'_1, \dotsc, v'_{b_{m-1}} \in \ker (A - \lambda E_n)^{m-1}$, so dass
        \begin{align*}
           &\,          \ker (A - \lambda E_n)^{m-1}  \\
          =&\,          \ker (A - \lambda E_n)^{m-2}
                \oplus  \Lin( (A - \lambda E_n) v_1, \dotsc, (A - \lambda E_n) v_{b_m} )  \\
           &\,          \phantom{\ker (A - \lambda E_n)^{m-2}\,}
                \oplus  \Lin( v'_1, \dotsc, v'_{b_{m-1}}  )
        \end{align*}
        gilt.
      \item
        Hierdurch erhält man für $B_\lambda$ die weitere Basisvektoren
        \begin{align*}
          (A - \lambda E_n)^{m-2} v'_1\,,\,         &\dotsc,\,  (A - \lambda E_n) v'_1\,,\,         v'_1\,,     \\
          (A - \lambda E_n)^{m-2} v'_2\,,\,         &\dotsc,\,  (A - \lambda E_n) v'_2\,,\,         v'_2\,,     \\
                                                    &\dotsc,                                                    \\
          (A - \lambda E_n)^{m-2} v'_{b_{m-1}}\,,\, &\dotsc,\,  (A - \lambda E_n) v'_{b_{m-1}}\,,\, v'_{b_{m-1}}\,.
        \end{align*}
        Hierbei bilden die Basisvektoren aus derselben Zeile jeweils eine Jordankette zum Eigenwert $\lambda$ von Länge $m-1$.
        (Hierfür wähle man etwa eine Basis $u_1, \dotsc, u_k$ von $\ker (A - \lambda E_n)^{m-2}$ und ergänze die Familie
        \[
          \bigl(
            u_1 \,,\, \dotsc \,,\, u_k \,,\,
            (A - \lambda E_n) v_1 \,,\, \dotsc \,,\, (A - \lambda E_n) v_{b_m}
          \bigr)
        \]
        zu einer Basis von $\ker (A - \lambda E_n)^{m-1}$.
        Dann lassen sich $v'_1, \dotsc, v'_{b_{m-1}}$ als die neu hinzugekommenen Basisvektoren wählen.)
      \item
        Man wähle nun $v''_1, \dotsc, v''_{b_{m-2}} \in \ker (A - \lambda E_n)^{m-2}$, so dass
        \begin{align*}
           &\,          \ker (A - \lambda E_n)^{m-2}  \\
          =&\,          \ker (A - \lambda E_n)^{m-3}
                \oplus  \Lin( (A - \lambda E_n)^2 v_1, \dotsc, (A - \lambda E_n)^2 v_{b_m} )  \\
           &\,          \phantom{\ker (A - \lambda E_n)^{m-2}\,}
                \oplus  \Lin( (A - \lambda E_n) v'_1, \dotsc, (A - \lambda E_n) v'_{b_{m-1}}  ) \\
           &\,          \phantom{\ker (A - \lambda E_n)^{m-2}\,}
                \oplus  \Lin( v''_1, \dotsc, v''_{b_{m-2}}  )
        \end{align*}
        gilt.
        (Hierfür wähle man etwa eine Basis $u_1, \dotsc, u_k$ von $\ker (A - \lambda E_n)^{m-3}$ und ergänze die Familie
        \begin{align*}
          \bigl(
            u_1 \,,\, &\dotsc \,,\, u_k \,,
            \\
            (A - \lambda E_n)^2 v_1 \,,\, &\dotsc \,,\, (A - \lambda E_n)^2 v_{b_m} \,,
            \\
            (A - \lambda E_n) v'_1 \,,\, &\dotsc \,,\, (A - \lambda E_n) v'_{b_{m-1}}
          \bigr)
        \end{align*}
        zu einer Basis von $\ker (A - \lambda E_n)^{m-2}$.
        Dann lassen sich $v''_1, \dotsc, v''_{b_{m-2}}$ als die neu hinzugekommenen Basisvektoren wählen.)
      \item
        Hierdurch erhält man für $B_\lambda$ die weitere Basisvektoren
        \begin{align*}
          (A - \lambda E_n)^{m-3} v''_1\,,\,
          &\dotsc\,,\,  (A - \lambda E_n) v''_1\,,\,
          v''_1\,,
          \\
          (A - \lambda E_n)^{m-3} v''_2\,,\,
          &\dotsc\,,\,
          (A - \lambda E_n) v''_2\,,\,
          v''_2\,,
          \\
          &\dotsc\,,
          \\
          (A - \lambda E_n)^{m-3} v''_{b_{m-2}}\,,\,
          &\dotsc\,,\,
          (A - \lambda E_n) v''_{b_{m-2}}\,,\,
          v''_{b_{m-2}} \,.
        \end{align*}
        Hierbei bilden die Basisvektoren aus derselben Zeile jeweils eine Jordankette zum Eigenwert $\lambda$ von Länge $m-2$.
    \end{itemize}
    Durch Weiterführen der obigen Schritte erhält man schließlich eine Basis $B_\lambda$ von $\hauptspace{A}{\lambda}$.
    
  \item
    Sind $\lambda_1, \dotsc, \lambda_t$ die paarweise verschiedenen Eigenwerte von $A$, so ergibt sich durch Hintereinanderlegen der Basen $B_{\lambda_1}, \dotsc, B_{\lambda_t}$ eine Basis $B$ von $K^n$.
    (Hier nutzen wir, dass $K^n = \hauptspace{A}{\lambda_1} \oplus \dotsb \oplus \hauptspace{A}{\lambda_t}$ gilt.)
 
  \item
    Die Basis $B$ ist eine Jordanbasis von $A$.
    Indem man die Basisvektoren von $B$ als Spalten in eine Matrix $S \in \matrices{n}{K}$ einträgt, erhält man schließlich eine invertierbare Matrix $S \in \GL{n}{K}$, so dass $S^{-1} A S$ in Jordan-Normalform ist.
    Dabei sind die Blöcke zunächst nach den Eigenwerten $\lambda_1, \dotsc, \lambda_t$ in der zuvor gewählten Reihenfolge sortiert, und die Blöcke zum gleichen Eigenwert $\lambda_i$ sind jeweils in absteigender Größe sortiert.
\end{itemize}

\begin{example}
  Für die Matrix
  \[
              A
    \defined \begin{pmatrix*}[r]
                 3  & 0 & 1 \\
                 1  & 2 & 1 \\
                -1  & 0 & 1
              \end{pmatrix*}
    \in \matrices{3}{\Complex}
  \]
  gilt $\charpol{A} = (X-2)^3$.
  Also ist $2$ der einzige Eigenwert von $A$.
  Es gilt
  \[
      \ker (A - 2E_n)
    = \ker
      \begin{pmatrix*}[r]
         1  & 0 &  1  \\
         1  & 0 &  1  \\
        -1  & 0 & -1
      \end{pmatrix*}
    = \Lin\left( \columnrvector{1 \\ 0 \\ -1}, \columnvector{0 \\ 1 \\ 0} \right).
  \]
  Somit gilt $\dim \ker (A - 2E_n) = 2$.
  Also gibt es zwei Jordanblöcke.
  Die Jor\-dan-Nor\-mal\-form von $A$ ist somit
  \[
    \begin{pmatrix}
      2 & 1 &   \\
        & 2 &   \\
        &   & 2
    \end{pmatrix}.
  \]
  Zur Berechnung einer Jordanbasis von $A$ gehen wir wie folgt vor:
  \begin{itemize}
    \item
      Wir benötigen einen Vektor $v_1 \in \ker (A - 2 E_n)^2$ mit
      \[
          \ker (A - 2 E_n)^2
        = \ker (A - 2 E_n) \oplus \Lin(v_1) \,,
      \]
      also einen Vektor $v_1 \in \Complex^3$ mit $v_1 \notin \ker (A - 2 E_n)$.
      Hierfür bietet sich etwa
      \[
                  v_1
        \defined \columnvector{1 \\ 0 \\ 0}
      \]
      an.
      Damit erhalten wir auch den weiteren Basisvektor
      \[
          (A - 2 E_n) v_1
        = \columnrvector{1 \\ 1 \\ -1}.
      \]
    \item
      Nun benötigen wir noch einen Vektor $v_2 \in \ker (A - 2 E_n)$ mit
      \[
          \ker (A - 2 E_n)
        = \Lin( (A - 2 E_n) v_1 ) \oplus \Lin(v_2)
        = \Lin\left( \columnrvector{1 \\ 1 \\ -1} \right) \oplus \Lin(v_2).
      \]
      Es gilt
      \[
          \ker (A - 2 E_n)
        = \Lin\left( \columnrvector{1 \\ 0 \\ -1}, \columnvector{0 \\ 1 \\ 0} \right)
        = \Lin\left( \columnrvector{1 \\ 1 \\ -1}, \columnvector{0 \\ 1 \\ 0} \right) \,,
      \]
      also lässt sich
      \[
                  v_2
        \defined \columnvector{0 \\ 1 \\ 0}
      \]
      wählen.
  \end{itemize}
  Insgesamt haben wir somit für $A$ die Jordanbasis
  \[
              B
    \defined \left(
                \columnrvector{1 \\ 1 \\ -1},
                \columnvector{1 \\ 0 \\ 0},
                \columnvector{0 \\ 1 \\ 0}
              \right)
  \]
  von $\Complex^3$.
  Für die entsprechende invertierbare Matrix
  \[
              S
    \defined \begin{pmatrix*}[r]
                 1 &  1 & 0 \\
                 1 &  0 & 1 \\
                -1 &  0 & 0
              \end{pmatrix*}
    \in \GL{3}{\Complex}
  \]
  gilt dann, dass
  \[
      S^{-1} A S
    = \begin{pmatrix}
        2 & 1 &   \\
          & 2 &   \\
          &   & 2
      \end{pmatrix}.
  \]
\end{example}

\begin{example}
  Es sei $V$ ein $K$-Vektorraum mit Basis $B = (v_1, \dotsc, v_7)$ und $f \colon V \to V$ der eindeutige Endomorphismus mit $f(v_1) = f(v_2) = v_5$, $f(v_3) = f(v_4) = v_6$, $f(v_5) = f(v_6) = v_7$ und $f(v_7) = 0$.
  \[
    \begin{tikzcd}
        v_1
        \arrow{dr}
      & {}
      & v_2
        \arrow{dl}
      & {}
      & v_3
        \arrow{dr}
      & {}
      & v_4
        \arrow{dl}
      \\
        {}
      & v_5
        \arrow{drr}
      & {}
      & {}
      & {}
      & v_6
        \arrow{dll}
      & {}
      \\
        {}
      & {}
      & {}
      & v_7
        \arrow{d}
      & {}
      & {}
      & {}
      \\
        {}
      & {}
      & {}
      & 0
      & {}
      & {}
      & {}
    \end{tikzcd}
  \]
  Es gilt $f^3(v_i) = 0$ für alle $i = 1, \dotsc, 7$, und deshalb $f^3 = 0$.
  Also ist der Endomorphismus $f$ nilpotent, und besitzt somit eine Jordan-Normalform;
  dabei ist $0$ der einzige auftretende Eigenwert.
  
  Für die darstellende Matrix
  \begin{gather*}
              A
    \defined \repmatrixendo{f}{B}
    =         \begin{pmatrix}
                0 &   &   &   &   &   &   \\
                  & 0 &   &   &   &   &   \\
                  &   & 0 &   &   &   &   \\
                  &   &   & 0 &   &   &   \\
                1 & 1 &   &   & 0 &   &   \\
                0 & 0 & 1 & 1 &   & 0 &   \\
                0 & 0 & 0 & 0 & 1 & 1 & 0
              \end{pmatrix}
  \shortintertext{gilt}
      A^2
    = \begin{pmatrix}
        0 &   &   &   &   &   &   \\
          & 0 &   &   &   &   &   \\
          &   & 0 &   &   &   &   \\
          &   &   & 0 &   &   &   \\
          &   &   &   & 0 &   &   \\
          &   &   &   &   & 0 &   \\
        1 & 1 & 1 & 1 & 0 & 0 & 0
      \end{pmatrix}
    \qquad\text{und}\qquad
      A^3
    = 0.
  \end{gather*}
  Somit gelten
  \begin{align*}
        \ker A\phantom{^1}
    &=  \Lin( e_1 - e_2, e_3 - e_4, e_5 - e_6, e_7 ),
    \\
        \ker A^2
    &=  \Lin( e_1 - e_2, e_2 - e_3, e_3 - e_4, e_5, e_6, e_7 ),
    \\
        \ker A^3
    &=  K^7.
  \end{align*}
  Die maximale auftretende Blöckgröße ist also $m = 3$.
  Mit
  \begin{align*}
    d_1 &= \dim \ker A^{\phantom{1}} = 4\,,   \\
    d_2 &= \dim \ker A^2 = 6\,, \\
    d_3 &= \dim \ker A^3 = 7
  \end{align*}
  erhalten wir für $b_k$, die Anzahl der Jordanblöcke von Größe $k$, dass
  \begin{align*}
    b_1 &= 2 d_1 - d_2 = 2 \,,        \\
    b_2 &= 2 d_2 - d_3 - d_1 = 1 \,,  \\
    b_3 &= 2 d_3 - d_4 - d_2 = 1
  \end{align*}
  gilt.
  Die Jordan-Normalform von $A$ (und damit von $f$) besteht also aus zwei Blöcken der Größe $1$, einem Block der Größe $2$ und einem Block der Größe $3$ (jeweils zum Eigenwert $0$);
  sie ist also
  \[
    \begin{pmatrix}
      0 & 1 &   &   &   &   &   \\
        & 0 & 1 &   &   &   &   \\
        &   & 0 &   &   &   &   \\
        &   &   & 0 & 1 &   &   \\
        &   &   &   & 0 &   &   \\
        &   &   &   &   & 0 &   \\
        &   &   &   &   &   & 0
    \end{pmatrix}.
  \]
  Wir bestimmen nun eine entsprechende Jordanbasis:
  \begin{itemize}
    \item
      Wir benötigen zunächst $w_1 \in \ker A^3 = K^7$ mit $K^7 = \ker A^2 \oplus \Lin(w_1)$.
      Hierfür muss nur $w_1 \notin \ker A^2$ gelten.
      Wir wählen $w_1 \defined e_1$.
      Dann erhalten wir auch die weiteren Basisvektoren $A w_1 = e_5$ und $A^2 w_1 = e_7$, und insgesamt die Jordankette $(A^2 w_1, A w_1, w_1) = (e_7, e_5, e_1)$.
    \item
      Als Nächstes benötigen wir einen Vektor $w_2 \in \ker A^2$ mit
      \begin{align*}
            \ker A^2
        &=          \ker A
            \oplus  \Lin(A w_1)
            \oplus  \Lin(w_2) \\
        &=          \Lin(e_1 - e_2, e_3 - e_4, e_5 - e_6, e_7)
            \oplus  \Lin(e_5)
            \oplus  \Lin(w_2).
      \end{align*}
      Wir müssen also die Familie $( e_1 - e_2, e_3 - e_4, e_5, e_5 - e_6, e_7 )$ zu einer Basis von $\ker A^2$ ergänzen.
      Da
      \[
          \Lin( e_1 - e_2, e_3 - e_4, e_5, e_5 - e_6, e_7 ) \\
        = \Lin( e_1 - e_2, e_3 - e_4, e_5, e_6, e_7 )
      \]
      gilt, können wir $w_2 \defined e_2 - e_3$ wählen.
      Damit erhalten wir außerdem den Basisvektoren $A w_2 = e_5 - e_6$, und ingesamt die Jordankette $(A w_2, w_2) = (e_5 - e_6, e_2 - e_3)$.
    \item
      Schließlich brauchen wir noch Vektoren $w_3, w_4 \in \ker A$ mit
      \begin{align*}
            \ker A
        &=          \Lin(A^2 w_1)
            \oplus  \Lin(A w_2)
            \oplus  \Lin(w_3, w_4)  \\
        &=          \Lin(e_7)
            \oplus  \Lin(e_5 - e_6)
            \oplus  \Lin(w_3, w_4).
      \end{align*}
      Wir müssen also die Familie $(e_5 - e_6, e_7)$ zu einer Basis von $\ker A$ ergänzen.
      Hierfür können wir $w_3 \defined e_1 - e_2$ und $w_4 \defined e_3 - e_4$ wählen.
  \end{itemize}

  Ingesamt haben wir somit die Basis
  \begin{align*}
        C
    &=  (A^2 w_1, A w_1, w_1, A w_2, w_2, w_3, w_4) \\
    &=  (e_7, e_5, e_1, e_5 - e_6, e_2 - e_3, e_1 - e_2, e_3 - e_4)
  \end{align*}
  bzw.\ die Basiswechselmatrix
  \[
              S
    \defined \begin{pmatrix*}[r]
                0 & 0 & 1 &  0  &  0  &  1  &  0  \\
                0 & 0 & 0 &  0  &  1  & -1  &  0  \\
                0 & 0 & 0 &  0  & -1  &  0  &  1  \\
                0 & 0 & 0 &  0  &  0  &  0  & -1  \\
                0 & 1 & 0 &  1  &  0  &  0  &  0  \\
                0 & 0 & 0 & -1  &  0  &  0  &  0  \\
                1 & 0 & 0 &  0  &  0  &  0  &  0
              \end{pmatrix*}
    \in       \GL{7}{K} \,,
  \]
  bezüglich der
  \[
      \repmatrixendo{f}{C}
    = S^{-1} A S
    = \begin{pmatrix}
        0 & 1 &   &   &   &   &   \\
          & 0 & 1 &   &   &   &   \\
          &   & 0 &   &   &   &   \\
          &   &   & 0 & 1 &   &   \\
          &   &   &   & 0 &   &   \\
          &   &   &   &   & 0 &   \\
          &   &   &   &   &   & 0
      \end{pmatrix}
  \]
  gilt.
\end{example}





\section*{Implizites Bestimmen der Jordan-Normaform}

Es sei $f \colon V \to V$ ein Endomorphismus eines endlichdimensionalen $K$-Vektorraums $V$, der eine Jordan-Normalform $J \in \matrices{n}{K}$ besitzt.
Wir fassen im Folgenden einige Eigenschaften zusammen, die man über $J$ aus Informationen über $f$ erhalten kann:
\begin{itemize}
  \item
    Für das charakteristische Polynom $\charpol{f}$ gilt $\deg \charpol{f} = n$.
  \item
    Für das charakteristische Polynom $\charpol{f} = \charpol{J} = (X - \lambda_1)^{n_1} \dotsm (X - \lambda_t)^{n_t}$ mit $\lambda_i \neq \lambda_j$ für $i \neq j$ gilt
    \[
        n_i
      = \dim \hauptspace{f}{\lambda_i}
      = \text{wie oft $\lambda_i$ auf der Diagonalen von $J$ steht}.
    \]
    Insbesondere sind $\lambda_1, \dotsc, \lambda_t$ die Diagonaleinträge von $J$ (bis auf Vielfachheit).
  \item
    Für alle $\lambda \in K$, $k \geq 0$ sei $d_k(\lambda) \defined \dim \ker(f - \lambda \id_V)^k$, und es sei $d_{-1}(\lambda) = 0$.
    Dann gilt
    \begin{align*}
       &\,  \text{Anzahl der Jordanblöcke zu $\lambda$ von Größe $k$ in $J$} \\
      =&\,  2 d_k(\lambda) - d_{k-1}(\lambda) - d_{k+1}(\lambda) \,.
    \end{align*}
  \item
    Für alle $\lambda \in K$ gilt
    \[
        \text{Anzahl der Jordanblöcke zu $\lambda$ in $J$}
      = \dim \underbrace{ \ker (f - \lambda \id_V) }_{= \eigenspace{f}{\lambda}}
      = n - \rank (f - \lambda \id_V) \,.
    \]
  \item
    Für das Minimalpolynom $\minpol{f} = \minpol{J} = (X - \lambda_1)^{m_1} \dotsm (X - \lambda_t)^{m_t}$ mit $\lambda_i \neq \lambda_j$ für $i \neq j$ gilt
    \[
        m_i
      = \text{maximale auftretende Größe eines Jordanblockes zu $\lambda_i$ in $J$}.
    \]
  \item
    Es sei allgemeiner $p \in K[X]$, $p \neq 0$ ein Polymom mit $p = (X - \lambda_1)^{m'_t} \dotsm (X - \lambda_t)^{m'_t}$, wobei $\lambda_i \neq \lambda_j$ für $i \neq j$ gilt.
    Dann gilt $\minpol{f} \divides p$, wodurch sich die folgenden beiden Restriktionen an $J$ ergeben:
    \begin{enumerate}
      \item
        Jeder Diagonaleintrag von $J$ kommt in $\lambda_1, \dotsc, \lambda_t$ vor.
      \item
        Die Jordanblöcke zu $\lambda_i$ in $J$ sind jeweils höchstens $m'_i$ groß.
    \end{enumerate}
  \item
    Für $\charpol{f} = (X - \lambda_1) \dotsm (X - \lambda_n)$ gilt
    \begin{align*}
      \tr(f)  &=  \lambda_1 + \dotsb + \lambda_n \,,  \\
      \det(f) &=  \lambda_1 \dotsm \lambda_n \,.
    \end{align*}
    Die Spur ist also die Summe, und die Determinante der Eigenwerte (also der Diagonaleinträge von $J$).
\end{itemize}

Hierdurch kann man für (ausreichend) kleine Matrizen bereits Aussagen über die Jordan-Nor\-mal\-form treffen, ohne die Matrix selbst zu kennen.

\begin{example}
  Es sei $A \in \matrices{6}{\Complex}$ mit $(A - 2E_n)^2 (A - 3E_n) = 0$ und $\tr A = 14$.
  Dann ergeben sich für die Jordan-Normalform von $A$ die folgenden Informationen:
  \begin{itemize}
    \item
      Die möglichen Eigenwerte von $A$ sind die Nullstellen des Polynoms $(X-2)^2 (X-3)$, also $2$ und $3$.
    \item
      Sind $\lambda_1, \dotsc, \lambda_n \in \{2, 3\}$ die Diagonaleinträge von der Jordan-Normalform (also die Eigenwerte von $A$), so ergibt sich aus $14 = \tr A = \lambda_1 + \dotsb + \lambda_n$, dass $2$ viermal auf der Diagonale vorkommt, und $3$ zweimal.
    \item
      Außerdem folgt aus $(A - 2E_n)^2 (A - 3E_n) = 0$, dass die Jordanblöcke zu $2$ höchstens Größe $2$ haben, und die Jordanblöcke zu $3$ alle Größe $1$ haben.
  \end{itemize}
  Damit ergeben sich bis auf Permutation der Jordanblöcke die folgenden drei möglichen Jordan-Normalformen:
  \[
    \begin{pmatrix}
      2 & 1 &   &   &   &   \\
        & 2 &   &   &   &   \\
        &   & 2 & 1 &   &   \\
        &   &   & 2 &   &   \\
        &   &   &   & 3 &   \\
        &   &   &   &   & 3
    \end{pmatrix},
    \quad
    \begin{pmatrix}
      2 & 1 &   &   &   &   \\
        & 2 &   &   &   &   \\
        &   & 2 &   &   &   \\
        &   &   & 2 &   &   \\
        &   &   &   & 3 &   \\
        &   &   &   &   & 3
    \end{pmatrix},
    \quad
    \begin{pmatrix}
      2 &   &   &   &   &   \\
        & 2 &   &   &   &   \\
        &   & 2 &   &   &   \\
        &   &   & 2 &   &   \\
        &   &   &   & 3 &   \\
        &   &   &   &   & 3
    \end{pmatrix}
  \]
  Das Minimalpolynom der ersten beiden Matrizen ist $(X-2)^2 (X-1)$, und das Minimalpolynom der dritten Matrix ist $(X-2)(X-3)$.
\end{example}


\end{document}
