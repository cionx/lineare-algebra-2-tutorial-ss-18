\section{}


Wir geben zwei alternative Lösungen für diese Aufgabe an.





\subsection*{Alternative 1}

\begin{lemma}
  \label{lemma: vandermonde}
  Ist $K$ ein Körper, so gilt für alle $x_1, \dotsc, x_n \in K$, dass
  \begin{equation}
  \label{equation: vandermonde}
      \det
      \begin{pmatrix}
        1         & 1         & \cdots  & 1         \\
        x_1       & x_2       & \cdots  & x_n       \\
        x_1^2     & x_2^2     & \cdots  & x_n^2     \\
        \vdots    & \vdots    & \ddots  & \vdots    \\
        x_1^{n-1} & x_2^{n-1} & \cdots  & x_n^{n-1}
      \end{pmatrix}
    = \prod_{1 \leq i < j \leq n} (x_j - x_i) \,.
  \end{equation}
\end{lemma}

Man bezeichnet hierbei die Matrix
\[
            V(x_1, \dotsc, x_n)
  \defined  \begin{pmatrix}
              1         & 1         & \cdots  & 1         \\
              x_1       & x_2       & \cdots  & x_n       \\
              x_1^2     & x_2^2     & \cdots  & x_n^2     \\
              \vdots    & \vdots    & \ddots  & \vdots    \\
              x_1^{n-1} & x_2^{n-1} & \cdots  & x_n^{n-1}
            \end{pmatrix}
  \in       \matrices{n}{K}
\]
als \emph{Vandermonde-Matrix}, und die Determinantenformel \eqref{equation: vandermonde} als \emph{Van\-der\-monde-De\-ter\-mi\-nan\-te}.
Lemma~\ref{lemma: vandermonde} sollte aus der Linearen~Algebra~I (Übungszettel~14) bekannt sein;
ein Beweis findet sich etwa in \emph{Lineare Algebra} von Gerd Fischer.

\begin{theorem}
  \label{theorem: sign formula}
  Für jedes $\sigma \in S_n$ gilt
  \[
      \sign(\sigma)
    = \prod_{1 \leq i < j \leq n} \frac{\sigma(j) - \sigma(i)}{j - i} \,.
  \]
\end{theorem}

\begin{proof}
  Es seien $x_1 \defined 1, \dotsc, x_n \defined n$.
  Es gilt zu zeigen, dass
  \[
      \sign(\sigma)
    = \frac{\prod_{1 \leq i < j \leq n} (x_{\sigma(j)} - x_{\sigma(i)})}{\prod_{1 \leq i < j \leq n} (x_j - x_i)}
  \]
  gilt.
  Dabei gelten
  \begin{align*}
        \prod_{1 \leq i < j \leq n} (x_j - x_i)
    &=  \det V(x_1, \dotsc, x_n) \,,
    \\
        \prod_{1 \leq i < j \leq n} (x_{\sigma(j)} - x_{\sigma(i)})
    &=  \det V(x_{\sigma(1)}, \dotsc, x_{\sigma(n)}) \,.
  \end{align*}
  Die Matrix $V(x_{\sigma(1)}, \dotsc, x_{\sigma(n)})$ entsteht dabei aus der Matrix $V(x_1, \dotsc, x_n)$ durch Vertauschen der Spalten:
  Die $j$-te Spalte von $V(x_{\sigma(1)}, \dotsc, x_{\sigma(n)})$ ist die $\sigma(j)$-te Spalte von $V(x_1, \dotsc, x_n)$.
  Nach Lemma~12.17 aus der Vorlesung gilt somit
  \[
      V(x_{\sigma(1)}, \dotsc, x_{\sigma(n)})
    = V(x_1, \dotsc, x_n) P_{\sigma} \,,
  \]
  und somit auch
  \begin{align*}
        \det V(x_{\sigma(1)}, \dotsc, x_{\sigma(n)})
    &=  \det( V(x_1, \dotsc, x_n) P_\sigma )  \\
    &=  \det V(x_1, \dotsc, x_n) \cdot \det P_{\sigma}  \\
    &=  \sign(\sigma) \det V(x_1, \dotsc, x_n) \,.
  \end{align*}
  Insgesamt erhalten wir somit, dass
  \[
      \frac{\prod_{1 \leq i < j \leq n} (x_{\sigma(j)} - x_{\sigma(i)})}{\prod_{1 \leq i < j \leq n} (x_j - x_i)}
    = \frac{\sign(\sigma) \det V(x_1, \dotsc, x_n)}{\det V(x_1, \dotsc, x_n)}
    = \sign(\sigma) \,.
  \]
\end{proof}





\subsection*{Alternative 2}

Für alle $\sigma \in S_n$ schreiben wir im Folgenden abkürzend
\[
            \sign'(\sigma)
  \defined  \prod_{1 \leq i < j \leq n} \frac{\sigma(j) - \sigma(i)}{j - i} \,.
\]
Wir zeigen, dass $\sign' = \sign$ gilt.



\subsubsection*{Multiplikativität von $\sign'$}

Wir zeigen zunächst, dass die Abbildung $\sign'$ multiplikativ ist.
Anstelle von Paaren $(i,j)$ mit $1 \leq i < j \leq n$ wollen wir hierfür im Folgenden allgemeiner über Indexmengen der folgenden Form aufmultiplizieren.
\begin{equation}
\label{equation: definition of T}
  \begin{tabular}{c}
    Es ist $T \subseteq \{1, \dotsc, n\} \times \{1, \dotsc, n\}$ eine Teilmenge, \\
    die für alle $1 \leq i,j \leq n$ mit $i \neq j$ genau eines \\
    der beiden Paare $(i,j)$, $(j,i)$ enthält.
  \end{tabular}
\end{equation}
Das wichtigste Beispiel einer solchen Menge ist
\[
            T_0
  \defined  \{ (i,j) \suchthat 1 \leq i < j \leq n \} \,.
\]
Man bemerke, dass also
\[
    \sign'(\sigma)
  = \prod_{(i,j) \in T_0} \frac{\sigma(j) - \sigma(i)}{j - i}
\]
gilt.


\begin{lemma}
  Es sei $T$ eine Menge, die \eqref{equation: definition of T} erfüllt.
  Dann gilt für alle $\sigma \in S_n$, dass
  \[
      \sign'(\sigma)
    = \prod_{(i,j) \in T} \frac{\sigma(j) - \sigma(i)}{j - i} \,.
  \]
\end{lemma}

\begin{proof}
  Wir bezeichnen den rechten Ausdruck abkürzend mit $\sign'_T(\sigma)$.
  Per Definition von $\sign'$ gilt $\sign' = \sign'_{T_0}$.
  
  Erfüllen $T_1, T_2$ die Bedingung~\eqref{equation: definition of T}, so gilt $\sigma'_{T_1} = \sigma'_{T_2}$, denn für alle $1 \leq i, j \leq n$ mit $i \neq j$ gilt
  \[
      \frac{\sigma(j) - \sigma(i)}{j - i}
    = \frac{\sigma(i) - \sigma(j)}{i - j} \,,
  \]
  weshalb die Produkte $\sigma'_{T_1}(\sigma)$ und $\sigma'_{T_2}(\sigma)$ die gleichen Faktoren besitzen.
  
  Inbesondere gilt somit für die gegebene Menge $T$, dass $\sign'_{T} = \sign'_{T_0} = \sign'$.
\end{proof}

\begin{lemma}
  Erfüllt $T$ die Bedingung~\eqref{equation: definition of T}, so erfüllt für jedes $\sigma \in S_n$ auch die Menge
  \[
              \sigma(T)
    \defined  \{(\sigma(i), \sigma(j)) \suchthat (i,j) \in T\}
  \]
  die Bedingung~\eqref{equation: definition of T}.
\end{lemma}

\begin{proof}
  Dies folgt direkt aus der Bijektivität von $\sigma$.
\end{proof}

\begin{corollary}
\label{corollary: sign' is multiplicative}
  Für alle $\sigma, \tau \in S_n$ gilt $\sign'(\sigma \tau) = \sign'(\sigma) \sign'(\tau)$.
\end{corollary}

\begin{proof}
  Es gilt
  \begin{align*}
        \sign'(\sigma \tau)
    &=  \prod_{1 \leq i < j \leq n} \frac{(\sigma \tau)(j) - (\sigma \tau)(i)}{j - i}
     =  \prod_{1 \leq i < j \leq n} \frac{\sigma(\tau(j)) - \sigma(\tau(i))}{j - i} \\
    &=  \prod_{1 \leq i < j \leq n} \frac{\sigma(\tau(j)) - \sigma(\tau(i))}{\tau(j) - \tau(i)}
        \prod_{1 \leq i < j \leq n} \frac{\tau(j) - \tau(i)}{j - i} \\
    &=  \prod_{(i,j) \in \tau(T_0)} \frac{\sigma(j) - \sigma(i)}{j- i}
        \prod_{(i,j) \in T_0} \frac{\tau(j) - \tau(i)}{j - i}
     =  \sign'(\sigma) \sign'(\tau) \,.
  \end{align*}
\end{proof}



\subsubsection*{Bestimmung von $\sign'(\sigma_k)$}

Als Nächstes bestimmen wir $\sign'(\sigma)$ für Permuationen der Form $\sigma = \sigma_k$, d.h.\ für sogennante \emph{einfache Transpositionen}.

\begin{lemma}
\label{lemma: sign' is only a sign}
  Für alle $\sigma \in S_n$ gilt $\sign'(\sigma) = \pm 1$.
\end{lemma}

\begin{proof}
  Die beiden Produkte
  \begin{gather*}
      \prod_{1 \leq i < j \leq n} (j-i)
    = \prod_{(i,j) \in T_0} (j-i)
  \shortintertext{und}
      \prod_{1 \leq i < j \leq n} (\sigma(j) - \sigma(i))
    = \prod_{(i,j) \in T_0} (\sigma(j) - \sigma(i))
    = \prod_{(i,j) \in \sigma(T_0)} (j-i)
  \end{gather*}
  erhalten für alle $1 \leq i < j \leq n$ mit $i \neq j$ genau einen der beiden Faktoren $j-i$ oder $i-j$, und jeder der Faktoron ist auch von einer dieser Formen.
  Da sich die Faktoren $j-i$ und $i-j$ nur bis auf Vorzeichen unterscheiden, unterscheiden sich auch insgesamt die beiden Produkte nur bis auf Vorzeichen.
\end{proof}

\begin{corollary}
  Für alle $\sigma \in S_n$ gilt $\sign'(\sigma) = (-1)^{f(\sigma)}$, wobei $f(\sigma)$ die Anzahl der Paare $(i,j)$ mit $1 \leq i < j \leq n$ bezeichnet, für welche $\sigma(i) > \sigma(j)$ gilt.
\end{corollary}

\begin{proof}
  Die Zahl $f(\sigma)$ gibt genau die Anzahl der Faktoren
  \[
    \frac{\sigma(j) - \sigma(i)}{j - i}
  \]
  mit $1 \leq i < j \leq n$ an, welche negativ sind.
  Die Aussage folgt deshalb aus Lemma~\ref{lemma: sign' is only a sign}.
\end{proof}

\begin{remark}
  Man bezeichnet ein Paar $(i,j)$ mit $1 \leq i < j \leq n$ und $\sigma(j) > \sigma(i)$ als einen \emph{Fehlstand} von $\sigma$.
  Es ist also $f(\sigma)$ die Anzahl der Fehlstände von $\sigma$.
  Es lässt sich zeigen, dass
  \[
      f(\sigma)
    = \min
      \{
        r \in \Natural
      \suchthat
        \text{es gibt $1 \leq i_1, \dotsc, i_r < n$ mit $\sigma = \sigma_{i_1} \dotsm \sigma_{i_r}$}
      \}
  \]
  gilt.
  Man bezeichnet diese Zahl deshalb auch als \emph{Länge von $\sigma$}, und schreibt auch $\ell(\sigma)$ anstelle von $f(\sigma)$.
\end{remark}

\begin{lemma}
\label{lemma: sign' of simple transposition}
  Für alle $1 \leq k < n$ gilt $\sign'(\sigma_k) = -1$.
\end{lemma}

\begin{proof}
  Nach Lemma~\ref{lemma: sign' is only a sign} gilt es zu zeigen, dass $\sigma_k$ eine ungerade Anzahl an Fehlständen besitzt.
  Wir zeigen, dass $(k,k+1)$ der einzige Fehlstand von $\sigma_k$ ist.
  
  Gilt $i < k$, so gilt für alle $j > i$, dass auch $\sigma_k(j) > i$ gilt, denn die Menge
  \[
      \{
        1 \leq j' \leq n
        \suchthat
        j' > i
      \}
    = \{i+1, \dotsc, k, k+1, \dotsc, n\}
  \]
  wird von $\sigma_k$ wieder auf sich selbst abgebildet.
  Somit gilt in diesem Fall
  \[
      \sigma_k(j)
    > i
    = \sigma_k(i) \,.
  \]
  
  Gilt $j > k+1$, so ergibt sich analog, dass $\sigma_k(j) = j > \sigma_k(i)$ gilt.
  
  Gelten $i \geq k$ und $j \leq k+1$, so muss wegen $i < j$ bereits $i = k$, $j = k+1$ gelten.
  In diesem Fall gilt
  \[
      \sigma_k(j)
    = \sigma_k(k+1)
    = k
    < k+1
    = \sigma_k(k)
    = \sigma_k(i) \,.
  \]
  
  Ingesamt zeigt dies, dass $(k,k+1)$ der einzige Fehlstand von $\sigma_k$ ist.
\end{proof}



\subsection*{Beweis von $\sign' = \sign$}

\begin{corollary}
  Für alle $1 \leq i < r$ gilt $\sign'(\sigma_{i_1} \dotsm \sigma_{i_r}) = (-1)^r$.
\end{corollary}

\begin{proof}
  Dies folgt direkt aus Korollar~\ref{corollary: sign' is multiplicative} und Lemma~\ref{lemma: sign' of simple transposition}.
\end{proof}

\begin{proof}[Beweis von Theorem~\ref{theorem: sign formula}]
  Die Permutation $\sigma$ lässt sich als $\sigma = \sigma_{i_1} \dotsm \sigma_{i_r}$ schreiben.
  Deshalb gilt
  \[
      \sign(\sigma)
    = (-1)^r
    = \sign'(\sigma) \,.
    \qedhere
  \]
\end{proof}

\begin{remark}
  Es gibt nur zwei multiplikative Abbildungen $S_n \to \{1, -1\}$:
  Das Vorzeichen $\sign$ und die \enquote{triviale} Abbildung
  \[
            S_n
    \to     \{1, -1\} \,,
    \quad   \sigma
    \mapsto 1 \,.
  \]

\end{remark}











