\section*{Zu Universellen Eigenschaften \& Tensorprodukten}

Es sei $K$ ein Körper und es seien $V$ und $W$ zwei $K$-Vektorräume.
In der Vorlesung haben wir die universelle Eigenschaft des Tensorprodukts $V \otimes W$ kennengelernt.
Es gibt in der Mathematik eine Vielzahl an Objekten und Konstruktionen, die eine sogenannt universelle Eigenschaft haben.
Es gibt zwar keine (allgemein akzeptierte) Definition dafür, was genau eine \enquote{universelle Eigenschaft} ist, nützlich ist aber die folgende informelle Beschreibung:
 
\begin{idea}
  Es sei $X$ ein mathematisches Objekt eines bestimmten Typs ($K$-Vektorraum, Gruppe, Ring, metrischer Raum, topologischer Raum, usw.).
  Eine \emph{universelle Eigenschaft} von $X$ beschreibt entweder, wie für jedes andere andere Objekt $Y$ vom gleichen Typ die Homomorphismen $X \to Y$ aussehen, oder wie für jedes Objekt $Y$ vom gleichen Typ die Homomorphismen $Y \to X$ aussehen.
\end{idea}

Im Falle des Tensorprodukts $V \otimes W$ erklärt die universelle Eigeschaft, wie für jeden anderen $K$-Vektorraum $U$ die Homomorphismen $V \otimes W \to U$ aussehen:
Ist $f \colon V \otimes W \to U$ eine lineare Abbildung, so ist die Abbildung
\[
          f^\circ
  \colon  V \times W
  \to     U,
  \quad   (v,w)
  \mapsto f(v \otimes w)
\]
$K$-bilinear, und umgekert für jede $K$-bilineare Abbildung $\beta \colon V \times W \to U$ gibt es eine eindeutige lineare Abbildung $\induced{\beta} \colon V \otimes W \to U$ mit
\[
    \induced{\beta}(v \otimes w)
  = \beta(v,w)
\]
für alle $v \in V$, $w \in W$.
Damit ergibt sich eine Bijektion
\begin{align*}
                    \Hom_K(V,W)
  &\longrightarrow  \{ \text{bilineare Abbildungen $\beta \colon V \times W \to U$} \} \,,  \\
                    f
  &\longmapsto      f^\circ \,, \\
                    \induced{\beta}
  &\longmapsfrom    \beta \,.
\end{align*}
In diesem Sinne sind lineare Abbildungen $V \otimes W \to U$ \enquote{das gleiche} wie bilineare Abbildungen $V \times W \to U$.

Möchten wir also eine lineare Abbildung $f \colon V \otimes W \to U$ definieren, so müssen wir hierfür an das folgende Backrezept halten:

\begin{itemize}
  \item
    Wir müssen uns zunächst überlegen, worauf die Elementartensoren $v \otimes w$ mit $v \in V$, $w \in W$ abgebildet werden sollen, d.h.\ für welche Elemente $u_{v,w} \in U$ später $f(v \otimes w) = u_{v,w}$ gelten soll.
  \item
    Wir definieren nun die Abbildung
    \[
              \beta
      \colon  V \times W
      \to     U,
      \quad   (v,w)
      \mapsto u_{v,w} \,.
    \]
  \item
    Wir müssen nachrechnen, dass die Abbildung $\beta$ bilinear ist.
  \item
    Aus der universellen Eigenschaft des Tensorprodukts $V \otimes W$ folgt schließlich, dass es eine eindeutige lineare Abbildung $f \colon V \otimes W \to U$ gibt, so dass
    \[
        f(v \otimes w)
      = \beta(v,w)
      = u_{v,w}
    \]
    für alle $v \in V$, $w \in W$ gilt.
    Dies ist genau die gewünschte lineare Abbildung.
\end{itemize}

\begin{example}
  Es seien $V$ und $W$ zwei $K$-Vektorräume.
  \begin{itemize}
    \item
      Wir zeigen, dass es eine eindeutige lineare Abbildung $e \colon V^* \otimes V \to K$ mit
      \[
          e(\varphi \otimes v)
        = \varphi(v)
      \]
      für alle $\varphi \in V^*$, $v \in V$ gibt:
      
      Die Abbildung
      \[
                \tilde{e}
        \colon  V^* \times V
        \to     K,
        \quad   (\varphi, v)
        \mapsto \varphi(v)
      \]
      ist $K$-bilinear, denn für alle $\varphi, \varphi_1, \varphi_2 \in V^*$, $v, v_1, v_2 \in V$ und $\lambda \in K$ gilt
      \begin{gather*}
                \tilde{e}(\varphi_1 + \varphi_2, v)
        = (\varphi_1 + \varphi_2)(v)
        = \varphi_1(v) + \varphi_2(v)
        = \tilde{e}(\varphi_1, v) + \tilde{e}(\varphi_2, v) \,,
        \\
          \tilde{e}(\varphi, v_1 + v_2)
        = \varphi(v_1 + v_2)
        = \varphi(v_1) + \varphi(v_2)
        = \tilde{e}(\varphi, v_1) + \tilde{e}(\varphi, v_2) \,,
        \\
          \tilde{e}(\lambda \varphi, v)
        = (\lambda \varphi)(v)
        = \lambda \varphi(v)
        = \lambda \tilde{e}(\varphi, v) \,,
        \\
          \tilde{e}(\varphi, \lambda v)
        = \varphi(\lambda v)
        = \lambda \varphi(v)
        = \lambda \tilde{e}(\varphi, v) \,.
      \end{gather*}
      Nach der universellen Eigenschaft des Tensorprodukts induziert die bilineare Abbildung $\tilde{e}$ eine eindeutige lineare Abbildung $e \colon V \otimes V^* \to K$ mit
      \[
          e(\varphi \otimes v)
        = \tilde{e}(\varphi, v)
        = \varphi(v)
      \]
      für alle $\varphi \in V^*$, $v \in V$.
    \item
      Wir zeigen, dass es eine eindeutige lineare Abbildung $\sigma \colon V \otimes W \to W \otimes V$ mit
      \[
          \sigma(v \otimes w)
        = w \otimes v
      \]
      für alle $v \in V$, $w \in W$ gibt:
      
      Die Abbildung
      \[
                \tilde{\sigma}
        \colon  V \times W
        \to     W \otimes V
        \quad   (v,w)
        \mapsto w \otimes v
      \]
      ist $K$-bilinear, denn für alle $v, v_1, v_2 \in V$, $w, w_1, w_2 \in W$, $\lambda \in K$ gilt
      \begin{gather*}
          \tilde{\sigma}(v_1 + v_2, w)
        = w \otimes (v_1 + v_2)
        = w \otimes v_1 + w \otimes v_2
        = \tilde{\sigma}(v_1, w) + \tilde{\sigma}(v_2, w) \,,
        \\
          \tilde{\sigma}(v, w_1 + w_2)
        = (w_1 + w_2) \otimes v
        = w_1 \otimes v + w_2 \otimes v
        = \tilde{\sigma}(v, w_1) + \tilde{\sigma}(v, w_2) \,,
        \\
          \tilde{\sigma}(\lambda v, w)
        = w \otimes (\lambda v)
        = \lambda (w \otimes v)
        = \lambda \tilde{\sigma}(v, w) \,,
        \\
          \tilde{\sigma}(v, \lambda w)
        = (\lambda w) \otimes v
        = \lambda (w \otimes v)
        = \lambda \tilde{\sigma}(v, w) \,.
      \end{gather*}
      Nach der universellen Eigenschaft des Tensorprodukts induziert die bilineare Abbildung $\tilde{\sigma} \colon V \times W \to W \otimes V$ eine eindeutige lineare Abbildung $\sigma \colon V \otimes W \to W \otimes V$ mit
      \[
          \sigma(v \otimes w)
        = \tilde{\sigma}(v, w)
        = w \otimes v
      \]
      für alle $v \in V$, $w \in W$.
      
      Die lineare Abbildung $\sigma$ ist tatsächlich schon ein Isomorphismus:
      Für die eindeutige lineare Abbildung $\tau \colon W \otimes V \to V \otimes W$ mit
      \[
          \tau(w \otimes v)
        = v \otimes w
      \]
      für alle $w \in W$, $v \in V$ (deren Existenz wir oben gezeigt haben) gilt
      \[
          (\tau \circ \sigma)(v \otimes w)
        = \tau(\sigma(v \otimes w))
        = \tau(w \otimes v)
        = v \otimes w
      \]
      für alle $v \in V$, $w \in W$.
      Da $V \otimes W$ als $K$-Vektorraum von den Elementartensoren $v \otimes w$ mit $v \in V$, $w \in W$ erzeugt wird, folgt hieraus, dass bereits
      \[
          (\tau \circ \sigma)(x)
        = x
      \]
      für alle $x \in V \otimes W$ gilt, und somit $\tau \circ \sigma = \id_{V \otimes W}$.
      Durch Vertauschen der Rollen von $V$ und $W$ ergibt sich, dass auch $\sigma \circ \tau = \id_{W \otimes V}$ gilt.
  \end{itemize}
\end{example}

\begin{remark}
  Auch andere Konstruktionen, die wir bereits kennengelernt haben, besitzen universelle Eigenschaften, die wir bisher allerdings noch nicht kennen gelernt haben:
  \begin{enumerate}
    \item
      Für jede Familie $V_i$, $i \in I$ von $K$-Vektorräumen $V_i$ besitzt die direkte Summe $\bigoplus_{i \in I} V_i$ eine universelle Eigenschaft, durch die für jeden $K$-Vektorraum $W$ die linearen Abbildungen $\bigoplus_{i \in I} V_i \to W$ beschrieben werden.
    \item
      Für jede Familie $V_i$, $i \in I$ von $K$-Vektorräumen $V_i$ besitzt das Produkt $\prod_{i \in I} V_i$ eine universelle Eigenschaft, durch die für jeden $K$-Vektorraum $U$ die linearen Abbildungen $U \to \prod_{i \in I} V_i$ beschrieben werden.
    \item
      Für jede lineare Abbildung $f \colon V \to W$ zwischen $K$-Vektorräumen $V$ und $W$ besitzt der Kern $\ker(f)$ eine universelle Eigenschaft, durch den für jeden $K$-Vektorraum $U$ die linearen Abbildungen $U \to \ker(f)$ beschrieben werden.
    \item
      Für jeden $K$-Vektorraum $V$ und Untervektorraum $U \subseteq V$ besitzt der Quotientenvektorraum $V/U$ eine universelle Eigenschaft, durch die für jeden $K$-Vektorraum $W$ die linearen Abbildungen $V/U \to W$ beschrieben werden.
    \item
      Für jeden $K$-Vektorraum $V$ besitzt der Dualraum $V^*$ eine universelle Eigenschaft, durch die für jeden $K$-Vektorraum $U$ die linearen Abbildungen $U \to V^*$ beschrieben werden.
  \end{enumerate}
\end{remark}


