\section{}





\subsection{}

Die möglichen Smith-Normalenformen einer Matrix $A \in \matrices{3}{K}$ sind die Matrizen
\[
      \begin{pmatrix}
        p_1 &     &     \\
            & p_2 &     \\
            &     & p_3
      \end{pmatrix}
  \in \matrices{3}{K[X]}
\]
mit normierten Polynomen $p_1, p_2, p_3 \in K[X]$, für die $p_1 \divides p_2 \divides p_3$ gilt.
Dabei gilt $\charpol{A} = p_1 p_2 p_3$.

Die Matrix $A$ genau dann invertierbar, wenn $\ker(A) = 0$ gilt, wenn also $0$ kein Eigenwert von $A$ ist.
Dies ist genau dann der Fall wenn $X \notdivides \charpol{A} = p_1 p_2 p_3$ gilt, was wegen $p_1 \divides p_2 \divides p_3$ äquivalent zu $X \notdivides p_3$ ist.
Der konstante Koeffizient von $p_3$ darf also nicht verschwinden.

Aus $\charpol{A} = p_1 p_2 p_3$ und $\deg(p_1) \leq \deg(p_2) \leq \deg(p_3)$ folgt
\[
    3
  = \deg(\charpol{A})
  = \deg(p_1 p_2 p_3)
  = \deg(p_1) + \deg(p_2) + \deg(p_3) \,.
\]
Wir erhalten deshalb für die Grade der Polynome $p_1, p_2, p_3$ die folgenden drei Fälle:
\begin{itemize}
  \item
    Es gilt $\deg(p_1) = \deg(p_2) = \deg(p_3) = 1$.
  \item
    Es gilt $\deg(p_1) = 0$, $\deg(p_2) = 1$ und $\deg(p_3) = 2$.
  \item
    Es gilt $\deg(p_1) = \deg(p_2) = 0$ und $\deg(p_3) = 3$.
\end{itemize}



\subsubsection*{Der Fall $\deg(p_1) = \deg(p_2) = \deg(p_3) = 1$}

Zusammen mit den Teilbarkeitsrelationen $p_1 \divides p_2 \divides p_2$ ergibt sich in diesem Fall, dass bereits $p_1 = p_2 = p_3 = X - a$ für ein $a \in K$ gelten muss.
Die möglichen Smith-Normalenformen für $\matrices{3}{K}$ sind für diesen Fall also
\[
  \begin{pmatrix}
    X - a &       &       \\
          & X - a &       \\
          &       & X - a
  \end{pmatrix}
  \qquad\text{mit}\qquad
  a \in K \,.
\]
Die entsprechende Frobenius- und Weierstraß-Normalenformen sind jeweils
\[
  \begin{pmatrix}
    a &   &   \\
      & a &   \\
      &   & a
  \end{pmatrix}.
\]

Für $\matrices{3}{K}$ ergeben sich für diesen Fall also $\card{K} = 3$ Konjugationsklassen.
Für $\GL{3}{K}$ ergibt ist die zusätzliche Bedingung $X \notdivides (X - a)$ äquvalent zu $a \neq 0$, und es ergeben sich $\card{K} - 1 = 2$ Konjugationsklassen.



\subsubsection*{Der Fall $\deg(p_1) = 0$, $\deg(p_2) = 1$, $\deg(p_3) = 2$}

In diesem Fall ergibt sich, dass $p_1 = X - a$ für ein $a \in K$ gelten muss.
Aus $\deg(p_2) = 2$ und $p_1 \divides p_2$ ergibt sich, dass $p_2 = (X - b) p_1$ für ein $b \in K$ gilt.
Die möglichen Smith-Normalenformen für $\matrices{3}{K}$ sind für diesen Fall also
\[
  \begin{pmatrix}
    1 &       &                 \\
      & X - a &                 \\
      &       & (X - a)(X - b)
  \end{pmatrix}
  \qquad\text{mit}\qquad
  a, b \in K \,.
\]
Es gilt $(X - a)(X - b) = X^2 - (a+b) + ab$, die entsprechende Frobenius-Normalform ist also jeweils
\[
  \begin{pmatrix}
    a &   &     \\
      & 0 & -ab \\
      & 1 & a+b
  \end{pmatrix}.
\]
Die Weierstraß-Normalform hängt davon ab, wie das Polynom $(X - a)(X - b)$ in irreduzible Faktoren zerfällt:
Im Fall $a \neq b$ zerfällt es in die beiden verschiedenen Linearfaktoren $(X - a)$ und $(X - b)$;
im Fall $a = b$ gilt $(X - a)(X - b) = (X - a)^2$.
Somit ergeben sich die beiden möglichen Weierstraß-Normalenformen
\begin{gather*}
  \begin{pmatrix}
    a &   &   \\
      & a &   \\
      &   & b
  \end{pmatrix}
  \qquad\text{falls $a \neq b$}
\shortintertext{und}
  \begin{pmatrix}
    a &   &       \\
      & 0 & -a^2  \\
      & 1 &  2a
  \end{pmatrix}
  \qquad\text{falls $a = b$} \,.
\end{gather*}

Für $\matrices{3}{K}$ ergeben sich für diesen Fall also $\card{K} \cdot \card{K} = 3 \cdot 3 = 9$ Konjugationsklassen.
Für $\GL{3}{K}$ ergibt sich aus der zusätzlichen Bedingung $X \notdivides (X-a)(X-b)$, dass $a, b \neq 0$ gelten muss, wodurch sich nur $(\card{K} - 1) \cdot (\card{K} - 1) = 2 \cdot 2 = 4$ ergeben.



\subsubsection*{Der Fall $\deg(p_1) = \deg(p_2) = 0$, $\deg(p_3) = 3$}

Die möglichen Smith-Normalenformen für $\matrices{3}{K}$ sind für diesen Fall
\[
  \begin{pmatrix}
    1 &   &                       \\
      & 1 &                       \\
      &   & X^3 + a X^2 + b X + c
  \end{pmatrix}
  \qquad\text{mit}\qquad
  a, b, c \in K \,.
\]
Die zugehörige Frobenius-Normalform ist jeweils
\[
  \begin{pmatrix}
    0 & 0 & -c  \\
    1 & 0 & -b  \\
    0 & 1 & -a
  \end{pmatrix}.
\]
Die zugehörige Weierstraß-Normalform hängt allerdings stark davon ab, wie das Polynom $X^3 + a X^2 + b X + c$ in irreduzible Faktoren zerfällt.
Hierbei gibt es die folgenden Möglichkeiten:

\begin{itemize}
  \item
    Es gilt $X^3 + a X^2 + b X + c = (X - \alpha)^3$ für ein $\alpha \in K$.
    Die Weierstraß-Normalform  ist dann
    \[
      \begin{pmatrix}
        0 & 0 & -c  \\
        1 & 0 & -b  \\
        0 & 1 & -a
      \end{pmatrix}.
    \]
    Dies gilt für die Polynome
    \begin{align*}
      X^3     &= X^3      \,, \\
      X^3 + 2 &= (X-1)^3  \,, \\
      X^3 + 1 &= (X-2)^3  \,.
    \end{align*}
  \item
    Es gilt $X^3 + a X^2 + b X + c = (X - \alpha)^2 (X - \beta)$ für $\alpha, \beta \in K$ mit $\alpha \neq \beta$.
    Mit $(X - \alpha)^2 = X^2 - 2 \alpha + \alpha^2$ ergibt sich dann die Weierstraß-Normalform
    \[
      \begin{pmatrix}
        0 & -\alpha^2 &       \\
        1 &  2\alpha  &       \\
          &           & \beta
      \end{pmatrix}.
    \]
    Dies gilt für die Polynome
    \begin{align*}
      X^3 + 2 X^2           &=  X^2 (X - 1)   \,,   \\
      X^3 +   X^2           &=  X^2 (X - 2)   \,,   \\
      X^3 +   X^2 +   X     &=  (X-1)^2 X     \,,   \\
      X^3 + 2 X^2 + 2 X + 1 &=  (X-1)^2 (X-2) \,,   \\
      X^3 + 2 X^2 +   X     &=  (X-2)^2 X     \,,   \\
      X^3 +   X^2 + 2 X + 2 &=  (X-2)^2 (X-1) \,.
    \end{align*}
  \item
    Es gilt $X^3 + a X^2 + b X + c = (X - \alpha)(X - \beta)(X - \gamma)$ für paarweise verschiedene $\alpha, \beta, \gamma \in K$.
    Die Weierstraß-Normalform ist dann
    \[
      \begin{pmatrix}
        \alpha  &       &         \\
                & \beta &         \\
                &       & \gamma
      \end{pmatrix}.
    \]
    Das einzige solche Polynom ist
    \[
      X^3 + 2X = X(X-1)(X-2) \,.
    \]
  \item
    Es gilt $X^3 + a X^2 + b X + c = (X - \alpha)(X^2 + \beta X + \gamma)$ für $\alpha, \beta, \gamma \in K$, wobei der Faktor $X^2 + \beta X + \gamma$ irreduzibel ist.
    Die Weierstraß-Normalform ist dann
    \[
      \begin{pmatrix}
        \alpha  &   &         \\
                & 0 & -\gamma \\
                & 1 & -\beta
      \end{pmatrix}.
    \]
    Dies gilt für die Polynome
    \begin{align*}
      X^3         +   X     &=  X (X^2 + 1)           \,, \\
      X^3 +   X^2 + 2 X     &=  X (X^2 + X + 2)       \,, \\
      X^3 + 2 X^2 + 2 X     &=  X (X^2 + 2 X + 2)     \,, \\
      X^3 + 2 X^2 +   X + 2 &=  (X-1) (X^2 + 1)       \,, \\
      X^3         +   X + 1 &=  (X-1) (X^2 + X + 2)   \,, \\
      X^3 +   X^2 +     + 1 &=  (X-1) (X^2 + 2X + 2)  \,, \\
      X^3 +   X^2 +   X + 1 &=  (X-2) (X^2 + 1)       \,, \\
      X^3 + 2 X^2 +     + 2 &=  (X-2) (X^2 + X + 2)   \,, \\
      X^3         +   X + 2 &=  (X-2) (X^2 + 2X + 2)  \,.
    \end{align*}
  \item
    Das Polynom $X^3 + a X^2 + b X + c$ ist irreduzibel.
    Die Weierstraß-Normalform ist dann
    \[
      \begin{pmatrix}
        0 & 0 & -c  \\
        1 & 0 & -b  \\
        0 & 1 & -a
      \end{pmatrix}.
    \]
    Dies gilt für die Polyonme
    \begin{gather*}
      X^3         + 2 X + 1 \,,
      \qquad
      X^3         + 2 X + 2 \,,
      \qquad
      X^3 +   X^2       + 2 \,,
      \\
      X^3 +   X^2 +   X + 2 \,,
      \qquad
      X^3 +   X^2 + 2 X + 1 \,,
      \qquad
      X^3 + 2 X^2       + 1 \,,
      \\
      X^3 + 2 X^2 +   X + 1 \,,
      \qquad
      X^3 + 2 X^2 + 2 X + 2 \,.
    \end{gather*}
\end{itemize}

Für $\matrices{3}{K}$ ergeben sich für diesen Fall $\card{K} \cdot \card{K} \cdot \card{K} = 3 \cdot 3 \cdot 3 = 27$ Konjugationsklassen.
Für $\GL{3}{K}$ ist die Bedingung $X \notdivides (X^3 + a X^2 + b X + c)$ äquivalent zu $c \neq 0$, weshalb sich $\card{K} \cdot \card{K} \cdot (\card{K} - 1) = 3 \cdot 3 \cdot 3 \cdot 2 = 3 \cdot 3 \cdot 2 = 18$ Konjugationsklassen ergeben.



\subsubsection*{Gesamtzahl der Konjugationsklassen}

Ingesamt erhalten wir $3 + 9 + 27 = 39$ Konjugationsklassen für $\matrices{3}{K}$ und $2 + 4 + 18 = 24$ Konjugationsklassen für $\GL{3}{K}$.





\subsection{}

Es seien
\[
            A
  \defined  \begin{pmatrix*}[r]
               2  &  0  & 0 &  0  \\
              -1  &  1  & 0 &  0  \\
               0  & -1  & 0 & -1  \\
               1  &  1  & 1 &  2
            \end{pmatrix*}
  \quad\text{und}\quad
            B
  \defined  \begin{pmatrix*}[r]
              0 & 3 & 1 \\
              1 & 0 & 0 \\
              0 & 2 & 0
            \end{pmatrix*} \,.
\]
Wir erhalten die Smith-Normalform von $A$ durch
\begingroup
\allowdisplaybreaks
\begin{align*}
    M_X(A)
  =&\,
    \begin{pmatrix}
      X-2 &   0 &   0 &   0 \\
        1 & X-1 &   0 &   0 \\
        0 &   1 & X   &   1 \\
       -1 &  -1 &  -1 & X-2
    \end{pmatrix}
  \\
  \xrightarrow{Z_1 \leftrightarrow Z_2}&\,
    \begin{pmatrix}
        1 & X-1 &   0 &   0 \\
      X-2 &   0 &   0 &   0 \\
        0 &   1 & X   &   1 \\
       -1 &  -1 &  -1 & X-2
    \end{pmatrix}
  \\
  \xrightarrow{\substack{Z_2 \to Z_2 - (X-2) Z_1 \\ Z_3 \to Z_3 + Z_1}}&\,
    \begin{pmatrix}
      1 &         X-1 &   0 &   0 \\
      0 & -(X-2)(X-1) &   0 &   0 \\
      0 &           1 & X   &   1 \\
      0 &         X-2 &  -1 & X-2
    \end{pmatrix}
  \\
  \xrightarrow{S_2 \to S_2 - (X-1) S_1}&\,
    \begin{pmatrix}
      1 &           0 &   0 &   0 \\
      0 & -(X-2)(X-1) &   0 &   0 \\
      0 &           1 & X   &   1 \\
      0 &         X-2 &  -1 & X-2
    \end{pmatrix}
  \\
  \xrightarrow{Z_2 \leftrightarrow Z_3}&\,
    \begin{pmatrix}
      1 &           0 &   0 &   0 \\
      0 &           1 & X   &   1 \\
      0 & -(X-2)(X-1) &   0 &   0 \\
      0 &         X-2 &  -1 & X-2
    \end{pmatrix}
  \\
  \xrightarrow{\substack{Z_3 \to Z_3 + (X-2)(X-1)Z_2 \\ Z_4 \to Z_4 - (X-2) Z_2}}&\,
    \begin{pmatrix}
      1 & 0 &           0 &          0  \\
      0 & 1 &           X &          1  \\
      0 & 0 & X(X-2)(X-1) & (X-2)(X-1)  \\
      0 & 0 &   -1-X(X-2) &          0
    \end{pmatrix}
  \\
  \xrightarrow{\substack{S_3 \to S_3 - X S_2 \\ S_4 \to S_4 - S_2}}&\,
    \begin{pmatrix}
      1 & 0 &           0 &          0  \\
      0 & 1 &           0 &          0  \\
      0 & 0 & X(X-2)(X-1) & (X-2)(X-1)  \\
      0 & 0 &   -1-X(X-2) &          0
    \end{pmatrix}
  \\
  =&\,
    \begin{pmatrix}
      1 & 0 &           0 &          0  \\
      0 & 1 &           0 &          0  \\
      0 & 0 & X(X-2)(X-1) & (X-2)(X-1)  \\
      0 & 0 &    -(X-1)^2 &          0
    \end{pmatrix}
  \\
  \xrightarrow{Z_4 \to -Z_4}&\,
    \begin{pmatrix}
      1 & 0 &           0 &          0  \\
      0 & 1 &           0 &          0  \\
      0 & 0 & X(X-2)(X-1) & (X-2)(X-1)  \\
      0 & 0 &     (X-1)^2 &          0
    \end{pmatrix}
  \\
  \xrightarrow{Z_3 \leftrightarrow Z_4}&\,
    \begin{pmatrix}
      1 & 0 &           0 &          0  \\
      0 & 1 &           0 &          0  \\
      0 & 0 &     (X-1)^2 &          0  \\
      0 & 0 & X(X-2)(X-1) & (X-2)(X-1)
    \end{pmatrix}
  \\
  \xrightarrow{S_3 \to S_3 - X S_4}&\,
    \begin{pmatrix}
      1 & 0 &       0 &          0  \\
      0 & 1 &       0 &          0  \\
      0 & 0 & (X-1)^2 &          0  \\
      0 & 0 &       0 & (X-2)(X-1)
    \end{pmatrix}
  \\
  \xrightarrow{S_3 \to S_3 + S_4}&\,
    \begin{pmatrix}
      1 & 0 &           0 &          0  \\
      0 & 1 &           0 &          0  \\
      0 & 0 &     (X-1)^2 &          0  \\
      0 & 0 &  (X-2)(X-1) & (X-2)(X-1)
    \end{pmatrix}
  \\
  =&\,
    \begin{pmatrix}
      1 & 0 &           0 &          0  \\
      0 & 1 &           0 &          0  \\
      0 & 0 &  (X-1)(X-1) &          0  \\
      0 & 0 &  (X-2)(X-1) & (X-2)(X-1)
    \end{pmatrix}
  \\
  \xrightarrow{Z_3 \to Z_3 - Z_4}&\,
    \begin{pmatrix}
      1 & 0 &           0 &          0  \\
      0 & 1 &           0 &          0  \\
      0 & 0 &       (X-1) & -(X-2)(X-1) \\
      0 & 0 &  (X-2)(X-1) &  (X-2)(X-1)
    \end{pmatrix}
  \\
  \xrightarrow{Z_4 \to Z_4 - (X-2) Z_3}&\,
    \begin{pmatrix}
      1 & 0 &     0 &                             0 \\
      0 & 1 &     0 &                             0 \\
      0 & 0 & (X-1) &                   -(X-2)(X-1) \\
      0 & 0 &     0 &  (X-2)(X-1) + (X-2)(X-2)(X-1)
    \end{pmatrix}
  \\
  =&\,
    \begin{pmatrix}
      1 & 0 &     0 &               0 \\
      0 & 1 &     0 &               0 \\
      0 & 0 & (X-1) &     -(X-2)(X-1) \\
      0 & 0 &     0 & (X-1)(X-2)(X-1)
    \end{pmatrix}
  \\
  =&\,
    \begin{pmatrix}
      1 & 0 &     0 &             0 \\
      0 & 1 &     0 &             0 \\
      0 & 0 & (X-1) &   -(X-2)(X-1) \\
      0 & 0 &     0 & (X-1)^2 (X-2)
    \end{pmatrix}
  \\
  \xrightarrow{S_4 \to S_4 + (X-2) S_3}&\,
    \begin{pmatrix}
      1 & 0 &     0 &             0 \\
      0 & 1 &     0 &             0 \\
      0 & 0 & (X-1) &             0 \\
      0 & 0 &     0 & (X-1)^2 (X-2)
    \end{pmatrix}.
\end{align*}
\endgroup
Die Frobenius-Normalform von $A$ ist nun
\begin{align*}
   &\,  \begin{pmatrix}
          B(X-1)  &                               \\
                  & B\left( (X-1)^2 (X-2) \right)
        \end{pmatrix}
  \\
  =&\,  \begin{pmatrix}
          B(X-1)  &                                     \\
                  & B\left( X^3 - 4X^2 + 5X - 2 \right)
        \end{pmatrix}
  =     \begin{pmatrix*}[r]
          1 &   &   &     \\
            & 0 & 0 &  2  \\
            & 1 & 0 & -5  \\
            & 0 & 1 &  4
        \end{pmatrix*} \,,
\end{align*}
und die Weierstraß-Normalform von $A$ ist
\begin{align*}
        &\,   \begin{pmatrix}
                B(X-1)  &                         &         \\
                        & B\left( (X-1)^2 \right) &         \\
                        &                         & B(X-2)
              \end{pmatrix}
  \\
  \approx&\,  \begin{pmatrix}
                B\left( (X-1)^2 \right) &         &         \\
                                        & B(X-1)  &         \\
                                        &         & B(X-2)
              \end{pmatrix}
  \\
  =&\,        \begin{pmatrix}
                B\left( X^2 - 2X + 1 \right)  &         &         \\
                                              & B(X-1)  &         \\
                                              &         & B(X-2)
              \end{pmatrix}
  =           \begin{pmatrix*}[r]
                0 & -1 &   &    \\
                1 &  2 &   &    \\
                  &    & 1 &    \\
                  &    &   & 2
              \end{pmatrix*}.
\end{align*}
Wir erhalten die Smith-Normalform von $B$ durch
\begin{align*}
    M_X(B)
  =&\,
    \begin{pmatrix}
       X & -3 & -1  \\
      -1 &  X &  0  \\
       0 & -2 &  X
    \end{pmatrix}
  \xrightarrow{Z_1 \leftrightarrow Z_2}
    \begin{pmatrix}
      -1 &  X &  0  \\
       X & -3 & -1  \\
       0 & -2 &  X
    \end{pmatrix}
  \\
  \xrightarrow{Z_1 \leftrightarrow -Z_1}&\,
    \begin{pmatrix}
      1 &  -X &  0  \\
      X &  -3 & -1  \\
      0 &  -2 &  X
    \end{pmatrix}
  \xrightarrow{Z_2 \to Z_2 - X Z_1}
    \begin{pmatrix}
      1 &    -X &  0  \\
      0 & X^2-3 & -1  \\
      0 &    -2 &  X
    \end{pmatrix}
  \\
  \xrightarrow{S_2 \to S_2 + X S_1}&\,
    \begin{pmatrix}
      1 &     0 &  0  \\
      0 & X^2-3 & -1  \\
      0 &    -2 &  X
    \end{pmatrix}
  \xrightarrow{S_2 \leftrightarrow S_3}
    \begin{pmatrix}
      1 &  0  &     0 \\
      0 & -1  & X^2-3 \\
      0 &  X  &    -2
    \end{pmatrix}
  \\
  \xrightarrow{Z_2 \to -Z_2}&\,
    \begin{pmatrix}
      1 & 0 &      0  \\
      0 & 1 & -X^2+3  \\
      0 & X &     -2
    \end{pmatrix}
  \xrightarrow{Z_3 \to Z_3 - X Z_2}
    \begin{pmatrix}
      1 & 0 &        0  \\
      0 & 1 &   -X^2+3  \\
      0 & 0 & X^3-3X-2
    \end{pmatrix}
  \\
  \xrightarrow{S_3 \to S_3 - (-X^2+3) S_2}&\,
    \begin{pmatrix}
      1 & 0 &        0  \\
      0 & 1 &        0  \\
      0 & 0 & X^3-3X-2
    \end{pmatrix}
  =
    \begin{pmatrix}
      1 & 0 &            0  \\
      0 & 1 &            0  \\
      0 & 0 & (X-2)(X+1)^2
    \end{pmatrix}.
\end{align*}
Die Frobenius-Normalform von $A$ ist dementsprechend
\[
    B\left( X^3 - 3X - 2 \right)
  = \begin{pmatrix}
      0 & 0 & 2 \\
      1 & 0 & 3 \\
      0 & 1 & 0
    \end{pmatrix},
\]
und die Weierstraß-Normalform von $A$ ist
\[
    \begin{pmatrix}
      B(X-2)  &                         \\
              & B\left( (X+1)^2 \right)
    \end{pmatrix}
  = \begin{pmatrix}
      B(X-2)  &                         \\
              & B\left( X^2 + 2X + 1 \right)
    \end{pmatrix}
  = \begin{pmatrix*}[r]
      2 &   &     \\
        & 0 & -1  \\
        & 1 & -2
    \end{pmatrix*} \,.
\]




