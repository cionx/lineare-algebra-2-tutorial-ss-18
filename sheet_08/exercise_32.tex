\section{}





\subsection{}

\subsubsection*{(a) $\implies$ (c)}
Gilt $v + U = w + U$, so gilt $(v + U) \cap (w + U) = v + U \neq \emptyset$, denn wegen $U \neq \emptyset$ gilt auch $v + U \neq \emptyset$.

\subsubsection*{(c) $\implies$ (b)}
Gilt $(v + U) \cap (w + U) \neq \emptyset$, so gibt es $u_1, u_2 \in U$ mit $v + u_1 = w + u_2$.
Dann gilt $v - w = u_2 - u_1 \in U$.

\subsubsection*{(b) $\implies$ (c)}
Gilt $v - w \in U$, so gilt $(v - w) + U = U$ und deshalb $v + U = w + (v-w) + U = w + U$.





\subsection{}

Es handelt sich bei $\sim$ um eine Äquivalenzrelation:

\begin{itemize}
  \item
    Für alle $v \in V$ gilt $v + U = v + U$ und deshalb $v \sim v$.
  \item
    Für alle $v, v' \in V$ mit $v \sim v'$ gilt $v + U = v' + U$, also auch $v' + U = v + U$ und somit $v' \sim v$.
  \item
    Für alle $v, v', v'' \in V$ mit $v \sim v'$ und $v' \sim v''$ gelten $v + U = v' + U$ und $v' + U = v'' + U$, also gilt $v + U = v'' + U$ und somit $v \sim v''$.
\end{itemize}

\begin{remark}
  Ist $f \colon X \to Y$ eine Abbildung zwischen zwei Mengen $X$ und $Y$, so wird durch
  \[
          x \sim x'
    \iff  f(x) = f(x')
  \]
  eine Äquivalenzrelation auf $X$ definiert.
  Mit $X = V$, der Potenzmenge $Y = \mathcal{P}(V)$ und der Funktion $f \colon V \to \mathcal{P}(V)$ mit $f(v) \defined v + U$ ergibt sich die obige Äquivalenzrelation.
\end{remark}

Nach dem vorherigen Aufgabenteil lässt sich auch Bedingung (b) nutzen, um zu zeigen, dass $\sim$ eine Äquivalenzrelation ist:

\begin{itemize}
  \item
    Für alle $v \in V$ gilt $v - v \in 0$ und deshalb $v \sim v$.
  \item
    Für alle $v, v' \in V$ mit $v \sim v'$ gilt $v - v' \in U$, also auch $v' - v = -(v - v') \in U$ und somit $v' \sim v$.
  \item
    Für alle $v, v', v'' \in V$ mit $v \sim v'$ und $v' \sim v''$ gilt $v - v', v' - v'' \in U$, und deshalb auch $v - v'' = (v - v') + (v' - v'') \in U$, also $v \sim v''$.
\end{itemize}

Für jedes $v \in V$ gilt nun
\[
    \class{v}
  = \{ w \in V \suchthat w \sim v \}
  = \{ w \in V \suchthat w - v \in U \}
  = \{ w \in V \suchthat w \in v + U \}
  = v + U \,.
\]

\begin{remark}
  Man kann auch umgekehrt an diese beiden Aufgabenteile herangehen:
  Man zeigt zunächst, dass durch $v \sim w \iff v - w \in U$ eine Äquivalenzrelation auf $V$ definiert wird.
  Dann erhält man, dass $\class{v} = v + U$ für alle $v \in V$ gilt.
  Der erste Aufgabenteil ergibt sich dann daraus, dass
  \[
          \class{v} = \class{w}
    \iff  v \sim w
    \iff  \class{v} \cap \class{w} = \emptyset \,.
  \]
\end{remark}





\subsection{}

Wir zeigen zunächst die Wohldefiniertheit der Addition und Skalarmultiplikation:

\begin{itemize}
  \item
    Für alle $v, v', w, w' \in V$ mit $v \sim v'$ und $w \sim w'$ gilt $v - v', w - w' \in U$.
    Deshalb gilt auch
    \[
          (v + w) - (v' + w')
      =   (v - v') + (w - w')
      \in U \,,
    \]
    und somit $(v + w) \sim (v' + w')$.
  \item
    Für alle $v, v' \in V$ mit $v \sim v'$ gilt $v - v' \in U$, für alle $\lambda \in K$ deshalb auch
    \[
          \lambda v - \lambda v'
      =   \lambda (v - v')
      \in U \,,
    \]
    und somit $(\lambda v) \sim (\lambda v')$.
\end{itemize}
Die Vektorraum-Axiome lassen sich nun auf Repräsentanten nachrechnen:
\begin{itemize}
  \item
    Für alle $\class{v}, \class{v'}, \class{v''} \in V\!/U$ gilt
    \[
        (\class{v} + \class{v'}) + \class{v''}
      = \class{v + v'} + \class{v''}
      = \class{v + v' + v''}
      = \class{v} + \class{v' + v''}
      = \class{v} + (\class{v'} + \class{v''}) \,.
    \]
    Das zeigt die Assoziativität der Addition von $V\!/U$.
  \item
    Für alle $\class{v}, \class{v'} \in V\!/U$ gilt
    \[
        \class{v} + \class{v'}
      = \class{v + v'}
      = \class{v' + v}
      = \class{v'} + \class{v} \,.
    \]
    Das zeigt die Kommutativität der Addition von $V\!/U$.
  \item
    Für alle $\class{v} \in V\!/U$ gilt
    \[
        \class{v} + \class{0}
      = \class{v + 0}
      = \class{v} \,.
    \]
    Das zeigt, dass $\class{0}$ neutral bezüglich der Additon von $V\!/U$ ist, dass also $0_{V\!/U} = [0]$ gilt.
  \item
    Für alle $\class{v} \in V\!/U$ gilt
    \[
        \class{v} + \class{-v}
      = \class{v + (-v)}
      = \class{0}
      = 0_{V\!/U} \,.
    \]
    Das zeigt, dass $\class{-v}$ additiv inverse zu $\class{v}$ ist, dass also $-\class{v} = \class{-v}$ gilt.
  \item
    Für alle $\class{v} \in V\!/U$ gilt
    \[
        1 \cdot \class{v}
      = \class{1 \cdot v}
      = \class{v} \,.
    \]
  \item
    Für alle $\lambda, \mu \in K$ und $\class{v} \in V\!/U$ gilt
    \[
        \lambda (\mu \class{v})
      = \lambda \class{\mu v}
      = \class{\lambda \mu v}
      = (\lambda \mu) \class{v} \,.
    \]
  \item
    Für alle $\lambda \in K$ und $\class{v}, \class{v'} \in V\!/U$ gilt
    \[
        \lambda (\class{v} + \class{v'})
      = \lambda \class{v + v'}
      = \class{\lambda (v + v')}
      = \class{\lambda v + \lambda v'}
      = \class{\lambda v} + \class{\lambda v'}
      = \lambda \class{v} + \lambda \class{v'} \,.
    \]
  \item
    Für alle $\lambda, \mu \in K$ und $\class{v} \in V\!/U$ gilt
    \[
        (\lambda + \mu) \class{v}
      = \class{(\lambda + \mu) v}
      = \class{\lambda v + \mu v}
      = \class{\lambda v} + \class{\mu v}
      = \lambda \class{v} + \mu \class{v} \,.
    \]
\end{itemize}
Man bemerke, dass die Vektorraumstruktur auf $V\!/U$ genau so definiert ist, dass die kanonische Projektion
\[
          p
  \colon  V
  \to     V\!/U
  \quad   v
  \mapsto \class{v}
\]
linear ist.



\subsection{}

Für alle $v, v' \in V$ mit $v \sim v'$ gilt $v - v' \in \ker(f)$, also
\[
    f(v) - f(v')
  = f(v - v')
  = 0
\]
und deshalb $f(v) = f(v')$.
Außerdem gilt $f(v) \in \im(f)$ für alle $v \in V$.
Es folgt, dass $f$ eine wohldefinierte Abbildung
\[
          \induced{f}
  \colon  V\!/U
  \to     \im(f) \,,
  \quad   \class{v}
  \mapsto f(v)
\]
induziert.
Für alle $\class{v}, \class{v'} \in V\!/U$ gilt
\[
    \induced{f}(\class{v} + \class{v'})
  = \induced{f}(\class{v + v'})
  = f(v + v')
  = f(v) + f(v')
  = \induced{f}(\class{v}) + \induced{f}(\class{v'}) \,,
\]
und für alle $\lambda \in K$ und $\class{v} \in V\!/U$ gilt
\[
    \induced{f}(\lambda \class{v})
  = \induced{f}(\class{\lambda v})
  = f(\lambda v)
  = \lambda f(v)
  = \lambda \induced{f}(\class{v}) \,.
\]
Das zeigt, dass $\induced{f}$ linear ist.
Die Abbildung $\induced{f}$ ist surjektiv, denn es gilt
\[
    \im(\induced{f})
  = \{ \induced{f}(\class{v}) \suchthat \class{v} \in V\!/U \}
  = \{ f(v) \suchthat v \in V \}
  = \im(f) \,.
\]
Die Abbildung $\induced{f}$ ist injektiv, denn für alle $\class{v}, \class{v'} \in V\!/U$ gilt
\[
        \induced{f}(\class{v})  = \induced{f}(\class{v'})
  \iff  f(v)  = f(v')
  \iff  v - v' \in \ker(f)
  \iff  \class{v} = \class{v'} \,;
\]
alternativ ergibt sich die Injektivität von $\induced{f}$ durch
\[
        \class{v} \in \ker(f)
  \iff  \induced{f}(\class{v}) = 0
  \iff  f(v) = 0
  \iff  v \in \ker(f)
  \iff  \class{v} = 0 \,.
\]

\begin{remark}
  Die kanonische Projection $p \colon V \to V\!/U$, $v \mapsto \class{v}$ ist ein Epimorphismus, die kanonische Inklusion $i \colon \im(f) \to W$, $w \mapsto w$ ist ein Monomorphismus, und es gilt $f = i \circ \induced{f} \circ p$.
  Wir können also jede lineare Abbildung $f$ in die Komposition eines Epimorphismus $p$, Isomorphismus $\induced{f}$ und Monomorphismus $i$ zerlegen.
  \[
    \begin{tikzcd}[sep = large]
        V
        \arrow{r}[above]{f}
        \arrow[twoheadrightarrow]{d}[left]{p}
      & W
      \\
        V\!/U
        \arrow{r}[above]{\sim}[below]{\induced{f}}
      & \im(f)
        \arrow[hook]{u}[right]{i}
    \end{tikzcd}
  \]
\end{remark}


\begin{remark}
  Ist allgemeiner $U \subseteq V$ ein Untervektorraum und $f \colon V \to W$ eine lineare Abbildung, so induziert $f$ genau dann eine wohldefinierte Abbildung
  \[
            \induced{f}
    \colon  V\!/U
    \to     W \,,
    \quad   \class{v}
    \to     f(v) \,,
  \]
  wenn $U \subseteq \ker(f)$ gilt, wenn also $\restrict{f}{U} = 0$ gilt.
  Die Abbildung $\induced{f}$ ist dann linear, und es gelten $\im(\induced{f}) = \im(f)$ und $\ker(\induced{f}) = \ker(f)/U$.
  Ist umgekehrt $g \colon V\!/U \to W$ eine lineare Abbildung, so ist $\hat{g} \colon V \to W$ mit $\hat{g}(v) = g(\class{v})$ eine lineare Abbildung mit $U \subseteq \ker(\hat{g})$.
  Dies liefert eine Bijektion
  \begin{align*}
                      \Hom_K(V/U,W)
    &\longrightarrow  \{ \text{lineare Abbildungen $f \colon V \to W$ mit $\restrict{f}{U}= 0$} \} \,,
    \\
                      g
    &\longmapsto      \hat{g} \,,
    \\
                      f
    &\longmapsfrom    \induced{f} \,,
  \end{align*}
  die sich grafisch wie folgt darstellen lässt:
  \[
    \begin{tikzcd}[sep = large]
        V
        \arrow{r}[above]{f = \hat{g}}
        \arrow{d}[left]{v \mapsto \class{v}}
      & W
      \\
        V\!/U
        \arrow{ru}[below right]{\induced{f} = g}
      & {}
    \end{tikzcd}
  \]

  Man bezeichnet dies als den \emph{Homomorphiesatz}, bzw.\ die \emph{universelle Eigenschaft des Quotientenvektorraums}.
  Diese universelle Eigenschaft erklärt uns, wie für jeden anderen $K$-Vektorraum die linearen Abbildungen $V\!/U \to W$ aussehen (siehe Idee~\ref{idea: universal properties}).
\end{remark}

\begin{remark}
  Wir wollen hier an die analogen Aussagen über Äquivalenzrelationen auf Mengen erinnern\footnote{
  Wir nutzen den Begriff \enquote{erinnern} hier im mathematischen Sinne:
  Sofern dem Leser/der Leserin die folgenden Aussagen bisher unbekannt waren, so sollte er/sie sich diese Resultate jetzt erarbeiten.}:
  
  Ist $\sim$ eine Äquivalenzrelation auf einer Menge $X$, so induziert eine Abbildung $f \colon X \to Y$ genau dann eine wohldefinierte Funktion
  \[
            \induced{f}
    \colon  (X/{\sim})
    \to     Y,
    \quad   \class{x}
    \mapsto f(x) \,,
  \]
  wenn $x \sim x' \implies f(x) = f(x')$ für alle $x, x' \in X$ gilt.
  Dabei gilt dann $\im(\induced{f}) = \im(f)$, und für alle $\class{x}, \class{x'} \in X/{\sim}$ gilt genau dann $\induced{f}(x) = \induced{f}(x')$, wenn $f(x) = f(x')$ gilt.
  
  Ist dabei $\sim$ die Äquivalenzrelation, die durch $x \sim x' \iff f(x) = f(x')$ gegeben ist, so ist die induziert Funktion $\induced{f} \colon (X/{\sim}) \to Y$ injektiv, und schränkt sich zu einer Bijektion $(X/{\sim}) \to \im(f)$ ein.
  
  Jede Funktion $f \colon X \to Y$ zwischen Mengen $X$ und $Y$ lässt sich für die Äquivalenzrelation $\sim$ auf $X$ mit $x \sim x' \iff f(x) = f(x')$ als $f = i \circ \induced{f} \circ p$ zerlegen, wobei $p \colon X \to X/{\sim}$, $x \to \class{x}$ die kanonische Projektion ist, und $i \colon \im(f) \to Y$, $y \to y$ die kanonische Inklusion.
  Es lässt sich also jede Funktion in die Komposition einer Surjektion $p$, Bijektion $\induced{f}$ und Injektion $i$ zerlegen:
  \[
    \begin{tikzcd}[sep = large]
        X
        \arrow{r}[above]{f}
        \arrow[twoheadrightarrow]{d}[left]{p}
      & Y
      \\
        X/{\sim}
        \arrow{r}[above]{\sim}[below]{\induced{f}}
      & \im(f)
        \arrow[hook]{u}[right]{i}
    \end{tikzcd}
  \]
\end{remark}





\subsection{}

Da $V = U \oplus U'$ gilt, lässt sich jedes $v \in V$ eindeutig als $v = u + u'$ mit $u \in U$ und $u' \in U'$ schreiben;
mit $f(v) \defined u'$ erhalten wir deshalb eine wohldefinierte Funktion $f \colon V \to U'$.

Die Funktion $f$ ist linear:
Haben $v_1, v_2 \in V$ die Zerlegungen $v_1 = u_1 + u'_1$ und $v_2 = u_2 + u'_2$ mit $u_1, u_2 \in U$ und $u'_1, u'_2 \in U'$, so gilt
\begin{gather*}
    v_1 + v_2
  = (u_1 + u'_1) + (u_2 + u'_2)
  = \underbrace{(u_1 + u_2)}_{\in U} + \underbrace{(u'_1 + u'_2)}_{\in U'}
\shortintertext{und deshalb}
    f(v_1 + v_2)
  = u'_1 + u'_2
  = f(v_1) + f(v_2) \,.
\end{gather*}
Hat $v \in V$ die Zerlegung $v = u + u'$ mit $u \in U$, $u' \in U'$, so gilt für alle $\lambda \in K$, dass
\begin{gather*}
    \lambda v
  = \lambda (u + u')
  = \underbrace{\lambda u}_{\in U} + \underbrace{\lambda u'}_{\in U'}
\shortintertext{und deshalb}
    f(\lambda v)
  = \lambda u'
  = \lambda f(u') \,.
\end{gather*}

Für alle $u' \in U$ gilt $f(u') = f(0 + u') = u'$, und somit gilt $\im(f) = U'$.
Hat $v \in V$ die Zerlegung $v = u + u'$ mit $u \in U$, $u' \in U'$, so gilt genau dann $v \in \ker(f)$, wenn $u' = f(v) = 0$, wenn also $v = u \in U$.
Deshalb gilt $\ker(f) = U$.

Mit den obigen Beobachtungen erhalten wir aus dem vorherigen Aufgabenteil, dass die lineare Abbildung $f \colon V \to U'$ einen Isomorphismus
\[
          \induced{f}
  \colon  V\!/U
  \to     U',
  \quad   \class{u + u'}
  \mapsto u'
\]
induziert.










