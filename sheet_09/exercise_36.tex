\section{}





\subsection{}

Für die Matrix
\[
            A
  \defined  \begin{pmatrix*}[r]
              0  & 3 &  4  \\
              -3  & 0 & -1  \\
              -4  & 1 &  0
            \end{pmatrix*}
  \in       \matrices{3}{\Real}
\]
gilt
\[
    s(v,w)
  = v^T A w
\]
für alle $v, w \in \Real^3$.
Hierdurch erkennen wir bereits, dass $s$ eine Bilinearform ist.
Heraus ergibt sich wiederum, dass wir $\overline{\,\cdot\,} = \pm \id_\Real$ wählen müssen, damit $s$ zu einer Metrik wird (siehe Lemma~17.4 in der Vorlesung).
Es gilt $A^T = -A$, also gilt
\[
    s(v,w)
  = v^T A w
  = (v^T A w)^T
  = w^T A^T v
  = - w^T A v
  = -s(w,v)
\]
für alle $v, w \in \Real^3$.
Wir müssen also $\overline{\,\cdot\,} = -\id_\Real$ wählen um $s$ zu einer Metrik zu machen.





\subsection{}

Die Metrik $s$ ist von Typ $(2^\circ)$ da $\overline{\,\cdot\,} = -\id_\Real$ gilt, und somit auch vom Typ $(2)$.
Da $s \neq 0$ und $\ringchar(\Real) \neq 2$ gilt kann $s$ somit nicht vom Typ $(1)$ sein, und somit auch nicht vom Typ $(1^\circ)$.





\subsection{}

Die Basis $B = (e_1, e_2, e_3)$ ist die Standardbasis von $\Real^3$, und durch direktes ablesen ergibt sich, dass
\[
  \repmatrixbilone{s}{B} = A \,.
\]
gil.
Außerdem gilt
\[
    \repmatrixhomo{\id_{\Real^3}}{C}{B}
  = \begin{pmatrix*}[r]
      0 & 0 &  1/4            \\
      0 & 1 &  1\phantom{/4}  \\
      1 & 0 & -3/4
    \end{pmatrix*} \,.
\]
Wir erhalten somit, dass
\begin{align*}
      \repmatrixbilone{s}{C}
  &=  \repmatrixhomo{\id_{\Real^3}}{C}{B}^T
      \repmatrixbilone{s}{B}
      \repmatrixhomo{\id_{\Real^3}}{C}{B}
  \\
  &=  \begin{pmatrix*}[r]
        0 & 0 &  1/4            \\
        0 & 1 &  1\phantom{/4}  \\
        1 & 0 & -3/4
      \end{pmatrix*}^T
      \begin{pmatrix*}[r]
         0  & 3 &  4  \\
        -3  & 0 & -1  \\
        -4  & 1 &  0
      \end{pmatrix*}
      \begin{pmatrix*}[r]
        0 & 0 &  1/4            \\
        0 & 1 &  1\phantom{/4}  \\
        1 & 0 & -3/4
      \end{pmatrix*}
  \\
  &=  \begin{pmatrix*}[l]
        0   & 0 &  \phantom{-}1   \\
        0   & 1 &  \phantom{-}0   \\
        1/4 & 1 &           -3/4
      \end{pmatrix*}
      \begin{pmatrix*}[r]
         0  & 3 &  4  \\
        -3  & 0 & -1  \\
        -4  & 1 &  0
      \end{pmatrix*}
      \begin{pmatrix*}[r]
        0 & 0 &  1/4            \\
        0 & 1 &  1\phantom{/4}  \\
        1 & 0 & -3/4
      \end{pmatrix*}
  \\
  &=  \begin{pmatrix*}[l]
        0   & 0 &  \phantom{-}1   \\
        0   & 1 &  \phantom{-}0   \\
        1/4 & 1 &           -3/4
      \end{pmatrix*}
      \begin{pmatrix*}[r]
         4  & 3 & 0 \\
        -1  & 0 & 0 \\
         0  & 1 & 0
      \end{pmatrix*}
   =  \begin{pmatrix*}[r]
         0  & 1 & 0 \\
        -1  & 0 & 0 \\
         0  & 0 & 0
      \end{pmatrix*}.
      \end{pmatrix*}.
\end{align*}
